\documentclass{beamer}
\usepackage{../tut-slides}
\usepackage{../mathoperatorsAuD}

\usepackage{csquotes}
\usepackage{cancel}

\usepackage{amsmath,amssymb}

\usepackage{tikz}
\usetikzlibrary{positioning,automata, matrix, trees}
\usetikzlibrary{calc,positioning,backgrounds,arrows.meta}
\usepackage{forest}


\usepackage{booktabs}
\usepackage{tabularx}
\usepackage{tabu}

\renewcommand{\tabularxcolumn}[1]{>{\hspace{0pt}}m{#1}}

\usepackage{listings}
\lstset{ 
	basicstyle=\footnotesize\ttfamily,        % the size of the fonts that are used for the code
	breakatwhitespace=false,         % sets if automatic breaks should only happen at whitespace
	breaklines=true,                 % sets automatic line breaking
	commentstyle=\itshape,    	     % comment style
	escapeinside={\%*}{*)},          % if you want to add LaTeX within your code
	extendedchars=true,              % lets you use non-ASCII characters; for 8-bits encodings only, does not work with UTF-8
	firstnumber=1,                % start line enumeration with line 1000
	frame=none,
	keywordstyle=\bfseries,       % keyword style
	morekeywords={}, 
	language=C,                 % the language of the code
	numbers=left,                    % where to put the line-numbers; possible: (none, left, right)
	numbersep=5pt,                   % how far the line-numbers are from the code
	numberstyle=\tiny\color{cdgray!50}, % the style that is used for the line-numbers
	rulecolor=\color{cddarkblue}, 
	tabsize=2,	                   % sets default tabsize to 2 spaces
}
\lstdefinestyle{am0}{ 
	basicstyle=\footnotesize\ttfamily,        % the size of the fonts that are used for the code
	breakatwhitespace=false,         % sets if automatic breaks should only happen at whitespace
	breaklines=true,                 % sets automatic line breaking
	commentstyle=\color{cdgray},    	     % comment style
	escapeinside={(*@}{@*)},          % if you want to add LaTeX within your code
	extendedchars=true,              % lets you use non-ASCII characters; for 8-bits encodings only, does not work with UTF-8
	firstnumber=1,                % start line enumeration with line 1000
	frame=none,
	keywordstyle=\bfseries,       % keyword style
	morekeywords={READ,LOAD,GT,JMC,STORE,JMP,WRITE}, 
%	language=AM0,                 % the language of the code
	numbers=left,                    % where to put the line-numbers; possible: (none, left, right)
	numbersep=5pt,                   % how far the line-numbers are from the code
	numberstyle=\tiny\ttfamily\color{cdgray!50}, % the style that is used for the line-numbers
	rulecolor=\color{cddarkblue}, 
	tabsize=2,	                   % sets default tabsize to 2 spaces
}


\usepackage{textgreek}


\renewcommand{\emph}[1]{\textbf{#1}}
\newcommand{\coloremph}[1]{\textcolor{cdpurple}{#1}}
\newcommand{\col}[1]{\textcolor{cdpurple}{\boldsymbol{#1}}}
\newcommand{\coll}[1]{\textcolor{cddarkgreen}{\boldsymbol{#1}}}
\newcommand{\colll}[1]{\textcolor{cdorange}{\boldsymbol{#1}}}
%\newcommand{\step}[2][]{\ensuremath{\overset{{#1} (\text{#2})}{=}}}
%\newcommand*{\astep}[2][]{\ensuremath{\overset{{#1} (\text{#2})}&{=}}}

\newcommand{\num}[1]{\ensuremath{\langle #1 \rangle}}

\undef\trans
\DeclareMathOperator{\trans}{trans}

\begin{document}	
	\title{Programmierung}
	\subtitle{Übung 10: C${}_\text{0}$ und abstrakte Maschine AM${}_\text{0}$}
	\author{Eric Kunze}
	\email{eric.kunze@mailbox.tu-dresden.de}
%	\city{TU Dresden}
	\date{}
%	\institute{Lehrstuhl für Grundlagen der Programmierung}
%	\titlegraphic{\includegraphics[width=2cm]{../TUD-white.pdf}}
	
	\maketitle
	

%%%%%%%%%%%%%%%%%%%%%%%%%%%%%%%%%%%%%%%%%%%%%%%%%%%%%%%%%%%%%%%%%%%%%%%%%%%%%

\begin{frame}[fragile] \frametitle{Inhalt}
	\begin{enumerate}
		\item Funktionale Programmierung
		\begin{enumerate}
			\item Einführung in Haskell: Listen
			\item Algebraische Datentypen
			\item Funktionen höherer Ordnung
			\item Typpolymorphie \& Unifikation
			\item Beweis von Programmeigenschaften
			\item \textlambda--Kalkül
		\end{enumerate}
		\item Logikprogrammierung
		\item Implementierung einer imperativen Programmiersprache
		\begin{enumerate}
			\item \textbf{Implementierung von C${}_\text{0}$}
			\item Implementierung von C${}_\text{1}$
		\end{enumerate}
		\item Verifikation von Programmeigenschaften
		\item H${}_\text{0}$ -- ein einfacher Kern von Haskell
	\end{enumerate}
\end{frame}



\section{Implementierung von C${}_\text{0}$ und abstrakte Maschine AM${}_\text{0}$}

\begin{frame}[fragile] \frametitle{$C_0$ und $AM_0$}
	\begin{itemize}
		\item \textbf{Ziel:} Implementierung einer einfachen Programmiersprache $C_1 \subset C$
		\pause
		\item \emph{Hier:} zunächst Einschränkung auf $C_0 \subset C_1$ 
		\begin{itemize}
			\item genau eine main-Funktion
			\item Zugriff auf \lstinline|stdio| durch \lstinline|#include|	
			\item einzig zugelassende Datenstruktur: \texttt{int}, Konstanten
			\item Kontrollstrukturen: Ein-/Ausgabebefehle, Zuweisungen, Sequenzen, Verzweigungen, bedingte Schleifen
		\end{itemize}
		\pause
		\item \emph{Implementierung} durch
		\begin{itemize}
			\item Syntax von $C_0$
			\item Befehle und Semantik einer abstrakten Maschine $AM_0$
			\item Übersetzer $C_0 \leftrightarrow AM_0$
		\end{itemize}
	\end{itemize}
\end{frame}


\begin{frame} \frametitle{Befehle und Semantik der $AM_0$}
	Wir bauen eine abstrakte Maschine $AM_0$, die unsere Berechnungen ausführen kann. Wir benötigen dafür:
	\begin{itemize}
		\item ein Ein- und Ausgabeband,
		\item einen Datenkeller,
		\item einen Hauptspeicher und 
		\item einen Befehlszähler
	\end{itemize}
	Nun müssen aber auch Aktionen ausgeführt werden, wie zum Beispiel das Einlesen vom Eingabeband in den Hauptspeicher. Dafür gibt es folgende Befehle:
\end{frame}

\begin{frame}
%	\begin{tikzpicture}[scale=1]
%	\foreach \i in {0,...,4} {
%		\draw[draw=black] (0,\i) rectangle ++(3,1) node[pos=.5] (h\i) {$h.\i$};
%		\draw[draw=black] (4,\i) rectangle ++(3,1) node[pos=.5] (d\i) {$d.\i$};
%	}
%	\node[below=of h0](HS){Hauptspeicher};
%	\node[below=of d0](DK){Datenkeller};
%	\end{tikzpicture}
\end{frame}

\begin{frame} \frametitle{Semantik der Befehle}
	Den Zustand der abstrakten Maschine beschreiben wir durch die Zustände der $5$ Komponenten, also als $5$-Tupel
	\begin{align*}
		&(m,d,h,inp,out)\\
		= \enskip &(\text{Befehlszähler}, \text{Datenkeller}, \text{Hauptspeicher}, \text{Input}, \text{Output})
	\end{align*}
	Jeder Befehl verändert den Zustand der Maschine -- er verändert also die Einträge in diesem Tupel.
	
	\begin{align*}
		\mathcal{C} [\![ \texttt{SUB} ]\!] &(m,d,h,inp,out) \defeq  \\
		&\text{if } d = d.1 : d.2 : \dots : d.n \\
		&\text{then } (m+1, (d.2 - d.1) : d.3 : \dots : d.n, inp, out)
	\end{align*}
\end{frame}

\begin{frame} \frametitle{Semantik der Befehle}
	\includegraphics[height=\textheight]{tut10-AM0Semantik.jpg}
\end{frame}


\begin{frame} \frametitle{Übersetzung von \texttt{if - then - else}}
	\small \centering
	\begin{align*}
	sttrans( \texttt{if (} &exp \texttt{)} \ stat_1 \ \texttt{else} \ stat_2, tab, a) := \\
	& boolexptrans(exp, tab) \\
	&\texttt{JMC} \ a.1 ; \\
	& sttrans(stat_1, tab, a.2) \\
	&\texttt{JMP} \ a.3; \\
	a.1: \quad & sttrans(stat_2, tab, a.4) \\
	a.3: \quad & \phantom{.}
	\end{align*}
	für alle $exp \in \mathrm{W}(\langle \mathrm{BoolExpression} \rangle )$, $stat_1, stat_2 \in \mathrm{W}(\langle \mathrm{Statement} \rangle )$, $tab \in \mathrm{Tab}$ und $a \in \mathbb{N}^\ast$.
\end{frame}





%%%%%%%%%%%%%%%%%%%%%%%%%%%%%%%%%%%%%%%%%%%%%%%%%%%%%%%%%%%%%%%%%%%%%%%%%%%%%%%%



\begin{frame}[fragile] \frametitle{Aufgabe 1}
	Wir betrachten das $C_0$-Programm $Max$:
	\begin{minipage}{\dimexpr0.5\linewidth-\fboxrule-\fboxsep}
		\begin{lstlisting}
#include <stdio.h>

int main( ) {
	int a, b, max;
	scanf("%i", &a);
	scanf("%i", &a);
		\end{lstlisting}
	\end{minipage}
	\begin{minipage}{\dimexpr0.5\linewidth-\fboxrule-\fboxsep}
		\begin{lstlisting}[firstnumber=7]
	if (a > b) 
		max = a;
	else max = b;
	printf("%d", max);
	return 0;
}
		\end{lstlisting}
	\end{minipage}

	\begin{itemize}
		\item[(a)] Berechnen Sie schrittweise das baumstrukturierte Programm $bMax_0 = \trans(Max)$ mit Hilfe der in der Vorlesung angegebenen Übersetzungsfunktionen. Dokumentieren Sie dabei jeden rekursiven Funktionsaufruf.
	\end{itemize}
\end{frame}

\begin{frame}[fragile, t] \frametitle{Aufgabe 1 -- Teil (a)}
	\small 
	\begin{minipage}[t]{\dimexpr0.5\linewidth-\fboxrule-\fboxsep}
		\textbf{Baumstrukturierte Adressen:}  \\
		
		\footnotesize
		\begin{tabular}{>{\ttfamily}r >{\ttfamily}l}
			& READ 1; \\
			& READ 2; \\
			& LOAD 1; \\
			& LOAD 2; \\
			& GT; \\
			& JMC 1.3.1; \\
			& LOAD 1; \\
			& STORE 3; \\
			& JMP 1.3.3; \\
			\textcolor{cdgray!50}{\tiny 1.3.1} & LOAD 2; \\
			& STORE 3; \\
			\textcolor{cdgray!50}{\tiny 1.3.3} & WRITE 3; 
		\end{tabular}
	\end{minipage}
	\pause \hfill
	\begin{minipage}[t]{\dimexpr0.4\linewidth-\fboxrule-\fboxsep}
		\textbf{Linearisierte Adressen:} \\
		\vspace{-8pt}
		\begin{lstlisting}[style=am0]
READ 1;
READ 2;
LOAD 1;
LOAD 2;
GT; 
JMC 10;
LOAD 1;
STORE 3;
JMP 12;
LOAD 2;
STORE 3;
WRITE 3; 
		\end{lstlisting}
	\end{minipage}
\end{frame}

\begin{frame} \frametitle{Aufgabe 1 -- Teil (b)}
	\small
	\emph{Ablauf der abstrakten Maschine:} 
	\begin{center}
		\begin{tabular}{rrcrclcrcrl}
			& BZ &,& DK &,& HS &,& Inp &,& Out & \\
			( & 1 &,& $\epsilon$ &,& [ ] &,& 5:7 &,& $\epsilon$ & ) \\
			( & 2 &,& $\epsilon$ &,& [1/5] &,& 7 &,& $\epsilon$ & ) \\
			( & 3 &,& $\epsilon$ &,& [1/5, 2/7] &,& $\epsilon$ &,& $\epsilon$ & ) \\
			( & 4 &,& 5 &,& [1/5, 2/7] &,& $\epsilon$ &,& $\epsilon$ & ) \\
			( & 5 &,& 7:5 &,& [1/5, 2/7] &,& $\epsilon$ &,& $\epsilon$ & ) \\
			( & 6 &,& 0 &,& [1/5, 2/7] &,& $\epsilon$ &,& $\epsilon$ & ) \\
			( & 10 &,& $\epsilon$ &,& [1/5, 2/7] &,& $\epsilon$ &,& $\epsilon$ & ) \\
			( & 11 &,& 7 &,& [1/5, 2/7 ] &,& $\epsilon$ &,& $\epsilon$ & ) \\
			( & 12 &,& $\epsilon$ &,& [1/5, 2/7, 3/7] &,& $\epsilon$ &,& $\epsilon$ & ) \\
			( & 13 &,& $\epsilon$ &,& [1/5, 2/7, 3/7] &,& $\epsilon$ &,& 7 & ) \\
		\end{tabular}
	\end{center}
	
	\pause   		
	
	\begin{equation*}
	\mathcal{P} [\![ Max_0 ]\!] (5:7) = proj_5^{(5)} \Bigl( \mathcal{I}[\![ Max_0 ]\!] (1,\epsilon, [],5:7,\epsilon) \Bigr) = 7
	\end{equation*}
\end{frame}
%
%%%%%%%%%%%%%%%%%%%%%%%%%%%%%%%%%%%%%%%%%%%%%%%%%%%%%%%%%%%%%%%%%%%%%%%%%%%%%%%%%

\begin{frame}[fragile] \frametitle{Aufgabe 2 -- Teil (a)}
	\begin{minipage}{\dimexpr0.5\linewidth-\fboxrule-\fboxsep}
		\begin{lstlisting}
#include <stdio.h> 

int main() { 
	int x1, x2;
	scanf("%i", &x1); 
	scanf("%i", &x2); 
	while (x1 > 0){
		\end{lstlisting}
	\end{minipage}
	\begin{minipage}{\dimexpr0.5\linewidth-\fboxrule-\fboxsep}
		\begin{lstlisting}[firstnumber=8]
		x1 = x2 - x1; 
		if (x2 > x1)
			x2 = x2 / 2;
	}
	printf("%d", x1); 
	return 0;
}
		\end{lstlisting}
	\end{minipage}

	\bigskip
	
	Übersetzen Sie das Programm mittels $\trans$ in $AM_0$-Code mit linearen Adressen. Geben Sie nur das Endergebnis der Übersetzung (keine Zwischenschritte) an!
\end{frame}

\begin{frame}[fragile] \frametitle{Aufgabe 2 -- Teil (a)}
	\begin{minipage}{\dimexpr0.25\linewidth-\fboxrule-\fboxsep}
		\begin{lstlisting}[firstnumber=1]
READ 1;
READ 2;
LOAD 1;
LIT 0;
GT;
		\end{lstlisting}
	\end{minipage}
	\begin{minipage}{\dimexpr0.25\linewidth-\fboxrule-\fboxsep}
		\begin{lstlisting}[firstnumber=6]
JMC 20;
LOAD 2;
LOAD 1;
SUB;
STORE 1;
		\end{lstlisting}
	\end{minipage}
	\begin{minipage}{\dimexpr0.25\linewidth-\fboxrule-\fboxsep}
		\begin{lstlisting}[firstnumber=11]
LOAD 2;
LOAD 1;
GT;
JMC 19;
LOAD 2;
		\end{lstlisting}
	\end{minipage}
	\begin{minipage}{\dimexpr0.25\linewidth-\fboxrule-\fboxsep}
		\begin{lstlisting}[firstnumber=16]
LIT 2;
DIV;
STORE 2;
JMP 3;
WRITE 1;
		\end{lstlisting}
	\end{minipage}
\end{frame}


\begin{frame}[fragile] \frametitle{Aufgabe 2 -- Teil (b)}
	\begin{minipage}{\dimexpr0.25\linewidth-\fboxrule-\fboxsep}
		\begin{lstlisting}[firstnumber=3]
LOAD 2;
LIT 5; 
LT;
		\end{lstlisting}
	\end{minipage}
	\begin{minipage}{\dimexpr0.25\linewidth-\fboxrule-\fboxsep}
		\begin{lstlisting}[firstnumber=6]
JMC 14; 
LOAD 1; 
LOAD 2;
		\end{lstlisting}
	\end{minipage}
	\begin{minipage}{\dimexpr0.25\linewidth-\fboxrule-\fboxsep}
		\begin{lstlisting}[firstnumber=9]
LIT 2;
MUL; 
ADD;
		\end{lstlisting}
	\end{minipage}
	\begin{minipage}{\dimexpr0.25\linewidth-\fboxrule-\fboxsep}
		\begin{lstlisting}[firstnumber=12]
STORE 2; 
JMP 3; 
WRITE 1;
		\end{lstlisting}
	\end{minipage}

	\bigskip 
	
	Erstellen Sie ein Ablaufprotokoll für dieses Programmfragment, bis die $AM_0$ terminiert. Die Startkonfiguration ist $(7, \epsilon, [1/3, 2/1], \epsilon, \epsilon)$.
\end{frame}

\begin{frame} \frametitle{Aufgabe 2 -- Teil (b)}
	\emph{Ablauf der abstrakten Maschine:} 
	\small
	\begin{center}
		\begin{tabular}{rrcrclcrcrl}
			& BZ &,& DK &,& HS &,& Inp &,& Out & \\
			( & 7 &,& $\epsilon$ &,& [1/3, 2/1] &,& $\epsilon$ &,& $\epsilon$ & ) \\
			( & 8 &,& 3 &,& [1/3, 2/1] &,& $\epsilon$ &,& $\epsilon$ & ) \\
			( & 9 &,& 1:3 &,& [1/3, 2/1] &,& $\epsilon$ &,& $\epsilon$ & ) \\
			( & 10 &,& 2:1:3 &,& [1/3, 2/1] &,& $\epsilon$ &,& $\epsilon$ & ) \\
			( & 11 &,& 2:3 &,& [1/3, 2/1] &,& $\epsilon$ &,& $\epsilon$ & ) \\
			( & 12 &,& 5 &,& [1/3, 2/1] &,& $\epsilon$ &,& $\epsilon$ & ) \\
			( & 13 &,& $\epsilon$ &,& [1/3, 2/5] &,& $\epsilon$ &,& $\epsilon$ & ) \\
			( & 3 &,& $\epsilon$ &,& [1/3, 2/5] &,& $\epsilon$ &,& $\epsilon$ & ) \\
			( & 4 &,& 5 &,& [1/3, 2/5] &,& $\epsilon$ &,& $\epsilon$ & ) \\
			( & 5 &,& 5:5 &,& [1/3, 2/5] &,& $\epsilon$ &,& $\epsilon$ & ) \\
			( & 6 &,& 0 &,& [1/3, 2/5] &,& $\epsilon$ &,& $\epsilon$ & ) \\
			( & 14 &,& $\epsilon$ &,& [1/3, 2/5] &,& $\epsilon$ &,& $\epsilon$ & ) \\
			( & 15 &,& $\epsilon$ &,& [1/3, 2/5] &,& $\epsilon$ &,& 3 & ) \\
		\end{tabular}
	\end{center}
\end{frame}

%\begin{frame}<handout:0> \frametitle{Aufgabe 2 -- Teil (b)}
%	\small
%	\begin{center}
%		\begin{tabular}{rrcrclcrcrl}
%			& BZ &,& DK &,& HS &,& Inp &,& Out & \\
%			( & 1 &,& $\epsilon$ &,& [ ] &,& 0:1 &,& $\epsilon$ & ) \\
%			( & 2 &,& $\epsilon$ &,& [1/0] &,& 1 &,& $\epsilon$ & ) \\
%			( & 3 &,& $\epsilon$ &,& [1/0, 2/1] &,& $\epsilon$ &,& $\epsilon$ & ) \\
%			( & 4 &,& 0 &,& [1/0, 2/1] &,& $\epsilon$ &,& $\epsilon$ & ) \\
%			( & 5 &,& 1:0 &,& [1/0, 2/1] &,& $\epsilon$ &,& $\epsilon$ & ) \\
%			( & 6 &,& 0:1:0 &,& [1/0, 2/1] &,& $\epsilon$ &,& $\epsilon$ & ) \\
%			( & 7 &,& 1:0 &,& [1/0, 2/1] &,& $\epsilon$ &,& $\epsilon$ & ) \\
%			( & 8 &,& 0 &,& [1/0, 2/1] &,& $\epsilon$ &,& $\epsilon$ & ) \\
%			( & 5 &,& 0 &,& [1/0, 2/1] &,& $\epsilon$ &,& $\epsilon$ & ) \\
%			( & 6 &,& 0:0 &,& [1/0, 2/1] &,& $\epsilon$ &,& $\epsilon$ & ) \\
%			( & 7 &,& 0 &,& [1/0, 2/1] &,& $\epsilon$ &,& $\epsilon$ & ) \\
%			( & 9 &,& $\epsilon$ &,& [1/0, 2/1] &,& $\epsilon$ &,& $\epsilon$ & ) \\
%			( & 10 &,& $\epsilon$ &,& [1/0, 2/1] &,& $\epsilon$ &,& 1 & ) \\
%		\end{tabular}
%	\end{center}
%\end{frame}
%
%\newcommand{\cw}[1]{\textcolor{cdgray!50}{#1}}
%\begin{frame} \frametitle{Aufgabe 2 -- Teil (b)}
%	\small
%	\begin{center}
%		\begin{tabular}{rrcrclcrcrl}
%			& BZ &,& DK &,& HS &,& Inp &,& Out & \\
%			( & 1 &,& $\epsilon$ &,& [ ] &,& 0:1 &,& $\epsilon$ & ) \\
%			( & 2 &,& \cw{$\epsilon$} &,& [1/0] &,& 1 &,& \cw{$\epsilon$} & ) \\
%			( & 3 &,& \cw{$\epsilon$} &,& [1/0, 2/1] &,& $\epsilon$ &,& \cw{$\epsilon$} & ) \\
%			( & 4 &,& 0 &,& \cw{[1/0, 2/1]} &,& \cw{$\epsilon$} &,& \cw{$\epsilon$} & ) \\
%			( & 5 &,& 1:0 &,& \cw{[1/0, 2/1]} &,& \cw{$\epsilon$} &,& \cw{$\epsilon$} & ) \\
%			( & 6 &,& 0:1:0 &,& \cw{[1/0, 2/1]} &,& \cw{$\epsilon$} &,& \cw{$\epsilon$} & ) \\
%			( & 7 &,& 1:0 &,& \cw{[1/0, 2/1]} &,& \cw{$\epsilon$} &,& \cw{$\epsilon$} & ) \\
%			( & 8 &,& 0 &,& \cw{[1/0, 2/1]} &,& \cw{$\epsilon$} &,& \cw{$\epsilon$} & ) \\
%			( & 5 &,& \cw{0} &,& \cw{[1/0, 2/1]} &,& \cw{$\epsilon$} &,& \cw{$\epsilon$} & ) \\
%			( & 6 &,& 0:0 &,& \cw{[1/0, 2/1]} &,& \cw{$\epsilon$} &,& \cw{$\epsilon$} & ) \\
%			( & 7 &,& 0 &,& \cw{[1/0, 2/1]} &,& \cw{$\epsilon$} &,& \cw{$\epsilon$} & ) \\
%			( & 9 &,& $\epsilon$ &,& \cw{[1/0, 2/1]} &,& \cw{$\epsilon$} &,& \cw{$\epsilon$} & ) \\
%			( & 10 &,& \cw{$\epsilon$} &,& \cw{[1/0, 2/1]} &,& \cw{$\epsilon$} &,& 1 & ) \\
%		\end{tabular}
%	\end{center}
%\end{frame}

\end{document}

