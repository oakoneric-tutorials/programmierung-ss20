\documentclass{beamer}
\usepackage{../tut-slides}
\usepackage{../mathoperatorsAuD}

\usepackage{csquotes}
\usepackage{cancel}

\usepackage{amsmath,amssymb}

\usepackage{tikz}
\usetikzlibrary{positioning,automata, matrix, trees}
\usetikzlibrary{calc,positioning,backgrounds,arrows.meta}
\usetikzlibrary{patterns,snakes}
\usepackage{forest}


\usepackage{booktabs}
\usepackage{tabularx}
\usepackage{tabu}
\newcommand*\head{\rowfont{\bfseries}}
\newcommand*{\tw}{\rowfont{\ttfamily}}

\renewcommand{\tabularxcolumn}[1]{>{\hspace{0pt}}m{#1}}

\usepackage{listings}
\lstset{ 
	basicstyle=\footnotesize\ttfamily,        % the size of the fonts that are used for the code
	breakatwhitespace=false,         % sets if automatic breaks should only happen at whitespace
	breaklines=true,                 % sets automatic line breaking
	commentstyle=\itshape,    	     % comment style
	escapeinside={\%*}{*)},          % if you want to add LaTeX within your code
	extendedchars=true,              % lets you use non-ASCII characters; for 8-bits encodings only, does not work with UTF-8
	firstnumber=1,                % start line enumeration with line 1000
	frame=none,
	keywordstyle=\bfseries,       % keyword style
	morekeywords={}, 
	language=C,                 % the language of the code
	numbers=left,                    % where to put the line-numbers; possible: (none, left, right)
	numbersep=5pt,                   % how far the line-numbers are from the code
	numberstyle=\tiny\color{cdgray!50}, % the style that is used for the line-numbers
	rulecolor=\color{cddarkblue}, 
	tabsize=2,	                   % sets default tabsize to 2 spaces
}
\lstdefinestyle{am0}{ 
	basicstyle=\footnotesize\ttfamily,        % the size of the fonts that are used for the code
	breakatwhitespace=false,         % sets if automatic breaks should only happen at whitespace
	breaklines=true,                 % sets automatic line breaking
	commentstyle=\color{cdgray},    	     % comment style
	escapeinside={(*@}{@*)},          % if you want to add LaTeX within your code
	extendedchars=true,              % lets you use non-ASCII characters; for 8-bits encodings only, does not work with UTF-8
	firstnumber=1,                % start line enumeration with line 1000
	frame=none,
	keywordstyle=\bfseries,       % keyword style
	morekeywords={READ,LOAD,GT,JMC,STORE,JMP,WRITE}, 
%	language=AM0,                 % the language of the code
	numbers=left,                    % where to put the line-numbers; possible: (none, left, right)
	numbersep=5pt,                   % how far the line-numbers are from the code
	numberstyle=\tiny\ttfamily\color{cdgray!50}, % the style that is used for the line-numbers
	rulecolor=\color{cddarkblue}, 
	tabsize=2,	                   % sets default tabsize to 2 spaces
}


\usepackage{textgreek}


\renewcommand{\emph}[1]{\textbf{#1}}
\newcommand{\coloremph}[1]{\textcolor{cdpurple}{#1}}
\newcommand{\col}[1]{\textcolor{cdpurple}{\boldsymbol{#1}}}
\newcommand{\coll}[1]{\textcolor{cddarkgreen}{\boldsymbol{#1}}}
\newcommand{\colll}[1]{\textcolor{cdorange}{\boldsymbol{#1}}}
%\newcommand{\step}[2][]{\ensuremath{\overset{{#1} (\text{#2})}{=}}}
%\newcommand*{\astep}[2][]{\ensuremath{\overset{{#1} (\text{#2})}&{=}}}

\newcommand{\num}[1]{\ensuremath{\langle #1 \rangle}}

\undef\trans
\DeclareMathOperator{\trans}{trans}

\newcommand{\logand}{ \ \land \ }

\begin{document}	
	\title{Programmierung}
	\subtitle{Übung 13: H${}_{\text{0}}$ -- ein einfacher Kern von Haskell}
	\author{Eric Kunze}
	\email{eric.kunze@mailbox.tu-dresden.de}
%	\city{TU Dresden}
	\date{}
%	\institute{Lehrstuhl für Grundlagen der Programmierung}
%	\titlegraphic{\includegraphics[width=2cm]{../TUD-white.pdf}}
	
	\maketitle
	

%%%%%%%%%%%%%%%%%%%%%%%%%%%%%%%%%%%%%%%%%%%%%%%%%%%%%%%%%%%%%%%%%%%%%%%%%%%%%

\begin{frame}[fragile] \frametitle{Inhalt}
	\begin{enumerate}
		\item Funktionale Programmierung
		\begin{enumerate}
			\item Einführung in Haskell: Listen
			\item Algebraische Datentypen
			\item Funktionen höherer Ordnung
			\item Typpolymorphie \& Unifikation
			\item Beweis von Programmeigenschaften
			\item \textlambda--Kalkül
		\end{enumerate}
		\item Logikprogrammierung
		\item Implementierung einer imperativen Programmiersprache
		\begin{enumerate}
			\item Implementierung von C${}_\text{0}$
			\item Implementierung von C${}_\text{1}$
		\end{enumerate}
		\item Verifikation von Programmeigenschaften
		\item \textbf{H${}_\text{0}$ -- ein einfacher Kern von Haskell}
	\end{enumerate}
\end{frame}


\section{H${}_\text{0}$ -- ein einfacher Kern von Haskell}

\begin{frame} \frametitle{H${}_\text{0}$}
	\begin{itemize}
		\item \textbf{Ziel:} verstehe den Zusammenhang H${}_\text{0}$ $\leftrightarrow$ AM${}_\text{0}$ $\leftrightarrow$ C${}_\text{0}$
		\item H${}_\text{0}$: \textit{tail recursive} Funktionen --- rechte Seite enthält
		\begin{itemize}
			\item keinen Funktionsaufruf
			\item einen Funktionsaufruf an der äußersten Stelle (nicht verschachtelt)
			\item eine Fallunterscheidung, deren Zweige wie oben aufgebaut sind
		\end{itemize}
	\end{itemize}
\end{frame}

\begin{frame} \frametitle{H${}_\text{0}$ $\leftrightarrow$ AM${}_\text{0}$}
	H${}_\text{0}$ ist klein genug, dass es auf der AM${}_\text{0}$ laufen kann:
	\begin{itemize}
		\item Befehle bleiben die gleichen
		\item baumstrukturierte Adressen beginnen mit Funktionsbezeichner (z.B. $f.1.3$)
		
		\medskip
		\item Übersetzung von rechten Seiten \texttt{... = \textit{exp}}: 
		\begin{itemize}
			\item Übersetze \texttt{exp}
			\item \texttt{STORE 1} \hspace{2em} (ja -- immer die \texttt{1})
			\item \texttt{WRITE 1}
			\item \texttt{JMP 0}
		\end{itemize}

		\medskip
		\item Übersetzung von Funktionsaufrufen \texttt{... = f x1 x2 x3}:
		\begin{itemize}
			\item \texttt{LOAD x1; LOAD x2; LOAD x3}
			\item \texttt{STORE x3; STORE x2; STORE x1} {\footnotesize(umgekehrte Reihenfolge!)}
			\item \texttt{JMP f}
		\end{itemize}
		
	\end{itemize}
\end{frame}

\begin{frame} \frametitle{H${}_\text{0}$ $\leftrightarrow$ C${}_\text{0}$}
	H${}_\text{0}$ (funktional) und C${}_\text{0}$ (imperativ) sind gleich stark -- wir können Programme jeweils ineinander äquivalent übersetzen!
	
	Standardisierung: 
	\begin{itemize}
		\item keine Konstanten
		\item Es gibt $m$ Variablen \texttt{x1, ..., xm} ($m \ge 1$)
		\item Wir lesen $k$ Variablen \texttt{x1, ..., xk} ein ($0 \le k \le m$)
		\item Es gibt genau eine Schreibanweisung direkt vor \texttt{return}
	\end{itemize}
\end{frame}

\begin{frame} \frametitle{C${}_\text{0}$ $\to$ H${}_\text{0}$}
	\begin{itemize}
		\item jedes Statement (in C${}_\text{0}$) erhält einen \textit{Ablaufpunkt}
		\item jeder Ablaufpunkt \texttt{i} wird durch eine Funktion \texttt{fi} (in H${}_\text{0}$) repräsentiert, die \textit{alle} Programmvariablen als Argumente hat
		\item Funktionswerte beschreiben Veränderungen im Programmablauf
	\end{itemize}

	(einfaches) \textbf{Beispiel:} 
	\begin{itemize}
		\item zwei Variablen \texttt{x1} und \texttt{x2}
		\item betrachte Zuweisung \texttt{x2 = x1 * x1} in C${}_\text{0}$
		\item Übersetzung zu \texttt{f x1 x2 = x1 (x1 * x1)}
	\end{itemize}
\end{frame}


\begin{frame} \frametitle{H${}_\text{0}$ $\to$ C${}_\text{0}$}
	Ein H${}_\text{0}$-Programm kann in C${}_\text{0}$ mittels \textit{einer} \texttt{while}-Schleife dargestellt werden. Dazu verwenden wir drei Hilfsvariablen:
	\begin{itemize}
		\item \texttt{flag} steuert den Ablauf der \texttt{while}-Schleife, d.h. wenn das H${}_\text{0}$-Programm terminiert, wird \texttt{flag} falsch
		\item \texttt{function} steuert in einer geschachtelten \texttt{if-then-else}-Anweisung, welche Funktion ausgeführt wird
		\item \texttt{result} speichert den Rückgabewert der Funktion
	\end{itemize}
\end{frame}


\section{Übungsblatt 13 \\ \textit{Aufgabe 1}}

\begin{frame}[t, fragile] \frametitle{Aufgabe 1 -- Teil (a)}
	\begin{equation*}
		f \colon \N \to \N \quad \mit \quad f(n) = \sum_{i=1}^{n} \prod_{j=1}^{i} j
	\end{equation*}
	\pause	
	\begin{lstlisting}[language=Haskell]
module Main where

--    i     sum     j    prod
f :: Int -> Int -> Int -> Int -> Int
f x1 x2 x3 x4
	= if x3 > 1
	  then f x1 x2 (x3 - 1) (x3 * x4)
	  else if x1 > 0
	       then f (x1 - 1) (x2 + x4) (x1 - 1) 1
	       else x2

main = do x1 <- readLn
       print (f x1 0 x1 1)
	\end{lstlisting}
\end{frame}

\begin{frame}[t, fragile] \frametitle{Aufgabe 1 -- Teil (b) \hfill H${}_\text{0}$ $\to$ AM${}_\text{0}$}
	
	\textbf{Gegeben:}
	\begin{lstlisting}[language=Haskell]
f :: Int -> Int
f x1 = if x1 < 42
       then x1
       else if x1 > 42
            then f (x1 `div` 2)
            else 42

	\end{lstlisting}
	
	\textbf{Gesucht:} äquivalentes AM${}_\text{0}$--Programm
	\pause
	\begin{lstlisting}[style=am0, numbers=none]
f:     LOAD 1; LIT 42; LT; JMC f.3;
       LOAD 1; STORE 1; WRITE 1; JMP 0;
f.3:   LOAD 1; LIT 42; GT; JMC f.2.3;
       LOAD 1; LIT 2; DIV; STORE 1; JMP f;
f.2.3: LIT 42; STORE 1; WRITE 1; JMP 0;
	\end{lstlisting}
\end{frame}

\begin{frame}[t, fragile] \frametitle{Aufgabe 1 -- Teil (c) \hfill H${}_\text{0}$ $\to$ C${}_\text{0}$}
	
	\textbf{Gegeben:}
	\begin{lstlisting}
f1  x1 = if ((x1 `mod` 2) == 0) then f11 x1
         else f12 x1
f11 x1 = f2 (x1 `div` 2)
f12 x1 = f2 (x1 - 1)
f2  x1 = f3 (2 * x1)
	\end{lstlisting}
	
	\textbf{Gesucht:} äquivalentes C${}_\text{0}$--Programm
	\pause
	\begin{lstlisting}[language=C]
if ((x1 % 2) == 0)
	x1 = x1 / 2;
else
	x1 = x1 - 1;
x1 = 2 * x1;
	\end{lstlisting}
\end{frame}





\section{Übungsblatt 13 \\ \itshape Aufgabe 2}

\begin{frame}[t, fragile] \frametitle{Aufgabe 2 -- Teil (a) \hfill H${}_\text{0}$ $\to$ AM${}_\text{0}$}
	\textbf{Gegeben:}
	\begin{lstlisting}
h :: Int -> Int -> Int -> Int
h x1 x2 x3 = if x3 > x1
             then (x2 - 1)
             else h x2 (x1 - x3) x2
	\end{lstlisting}
	
	\textbf{Gesucht:} äquivalentes AM${}_\text{0}$--Programm
	\pause
	\begin{lstlisting}[style=am0]
h:   LOAD 3; LOAD 1; GT; JMC h.3;
     LOAD 2; LIT 1; SUB; STORE 1; WRITE 1; JMP 0;
h.3: LOAD 2; LOAD 1; LOAD 3; SUB; LOAD 2;
     STORE 3; STORE 2; STORE 1; JMP h;
	\end{lstlisting}
\end{frame}

\begin{frame}[fragile] \frametitle{Aufgabe 2 -- Teil (b) \hfill H${}_\text{0}$ $\to$ C${}_\text{0}$}
	\textbf{Lösung:}
	
	\begin{minipage}[t]{\dimexpr0.5\linewidth-\fboxrule-\fboxsep}
		\begin{lstlisting}[numbers=none]
A:    scanf("%d", &x1);
      x1 = 3 + x1;
      x2 = 5;
      flag = 1;
      
B:    x2 == x1

C:    result = 30;
      flag = 0;
      
D:    result = x2;
      flag = 0;
		\end{lstlisting}
	\end{minipage}
	\begin{minipage}[t]{\dimexpr0.5\linewidth-\fboxrule-\fboxsep}
		\begin{lstlisting}[numbers=none]
E:    function == 2

F:    if (10 <= x2) {
            x1 = x1 - x2;
            x2 = x2 - 1;
      } else {
            x1 = x1 + x2;
            x2 = 10;
            function = 1;
      }
		\end{lstlisting}
	\end{minipage}


\end{frame}

\end{document}

