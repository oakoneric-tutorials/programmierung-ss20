\documentclass{beamer}
\usepackage{../tut-slides}
\usepackage{../mathoperatorsAuD}

\usepackage{csquotes}
\usepackage{cancel}

\usepackage{amsmath,amssymb}
\usepackage{enumerate}
\usepackage[inline]{enumitem} 		%customize label
\newcommand{\labelitemi}{\raisebox{1pt}{\scalebox{.9}{$\blacktriangleright$}}}
\newcommand{\labelitemii}{$\vartriangleright$}
\newcommand{\labelitemiii}{--}

\usepackage{booktabs}
\usepackage{tabularx}
\usepackage{tabu}
\newcommand*\head{\rowfont{\bfseries}}
\newcommand*{\tw}{\rowfont{\ttfamily}}

\renewcommand{\tabularxcolumn}[1]{>{\hspace{0pt}}m{#1}}

\usepackage{listings}
\lstset{ 
	basicstyle=\footnotesize\ttfamily,        % the size of the fonts that are used for the code
	breakatwhitespace=false,         % sets if automatic breaks should only happen at whitespace
	breaklines=true,                 % sets automatic line breaking
	commentstyle=\itshape,    	     % comment style
	escapeinside={\%*}{*)},          % if you want to add LaTeX within your code
	extendedchars=true,              % lets you use non-ASCII characters; for 8-bits encodings only, does not work with UTF-8
	firstnumber=1,                % start line enumeration with line 1000
	frame=single,
	keywordstyle=\bfseries,       % keyword style
	language=Haskell,                 % the language of the code
	numbers=left,                    % where to put the line-numbers; possible: (none, left, right)
	numbersep=5pt,                   % how far the line-numbers are from the code
	numberstyle=\tiny\color{cdgray!50}, % the style that is used for the line-numbers
	rulecolor=\color{cddarkblue}, 
	tabsize=2,	                   % sets default tabsize to 2 spaces
}

\begin{document}	
	\title{Programmierung}
	\subtitle{Übung 1: Einleitung}
	\author{Eric Kunze}
	\email{eric.kunze@mailbox.tu-dresden.de}
	\city{TU Dresden}
%	\institute{Lehrstuhl für Grundlagen der Programmierung}
	\titlegraphic{\includegraphics[width=2cm]{../TUD-white.pdf}}

	\maketitle
	
	\begin{frame} \frametitle{Für die vergesslichen}
		\begin{minipage}{\dimexpr0.75\linewidth-\fboxrule-\fboxsep}
			\textbf{Wer bin ich?}
			\begin{itemize}
				\item Eric [Kunze]
				\item \url{eric.kunze@mailbox.tu-dresden.de}
				\item Fragen, Wünsche, Vorschläge, \dots \\ gern jederzeit wie auch immer
			\end{itemize}
		\end{minipage}
		\begin{minipage}{\dimexpr0.25\linewidth-\fboxrule-\fboxsep}
			\includegraphics[width=3cm]{./tut01_pic.jpg}
		\end{minipage}
	\end{frame}

	\begin{frame} \frametitle{Das etwas andere Semester}
		\footnotesize
		\textbf{offizielles Angebot}
		\begin{itemize}
			\item Vorlesungen
			\begin{itemize}
				\item Lehrveranstaltungswebsite: \\ \url{www.orchid.inf.tu-dresden.de/teaching/2020ss/prog/}
				\item OPAL-Kurs: Link siehe LV-Website
			\end{itemize}
			\item Übungen
			\begin{itemize}
				\item selbstständige Bearbeitung der Übungen
				\item Abgabe via OPAL an (zufällig) zugewiesenen Tutor
				\item Rückgabe der korrigierten Arbeiten via OPAL
			\end{itemize} 
		\end{itemize}
		\textbf{mein (privates) Angebot}
		\begin{itemize}
			\item Übernahme der Korrekturen
			\item Vidcasting bzw. Live-Übungen
		\end{itemize}
	\end{frame}


	\begin{frame} \frametitle{Lösungen}
		\textbf{Slides werden mit Sourcecode auf Github zur Verfügung stehen.}
		\begin{itemize}[leftmargin=*]
			\item \url{https://github.com/oakoneric/programmierung-ss20}
			\item \url{github.com} $\to$ oakoneric $\to$ programmierung-ss20
			\item evtl. zusätzliche Materialien (nach Bedarf)
			\item \alert{\textbf{kein Anspruch auf Vollständigkeit \& Korrektheit}}
			\item gefundene Fehler melden - am besten per Request
		\end{itemize}
	\end{frame}

%%%%%%%%%%%%%%%%%%%%%%%%%%%%%%%%%%%%%%%%%%%%%%%%%%%%%%%%%%%%%%%%%%%%%%%%%%%%%

\section{Einführung in Haskell}

    \begin{frame}[fragile] \frametitle{Haskell \& funktionale Programmierung}
    	\footnotesize
        \fbox{Haskell $=$ funktionale Programmiersprache}
        
        Wir programmieren nicht \textit{wie} berechnet wird, sondern \textit{was} berechnet wird.
    	
    	\pause
        \medskip
        \textbf{Mathe:} Die Addition $n + m$ können wir auch als Funktion  
        \begin{equation*}
            + \colon \mathbb{N} \times \mathbb{N} \to \mathbb{N} \qquad
            n+m = 
            \begin{cases}
            n & m=0 \\
            1 + \Big( n + (m-1) \Big) &\text{sonst} \\
            \end{cases}
        \end{equation*}
        definieren. \\
       
        \medskip \pause
        
        \textbf{Informatik:} Wir können eine Funktion \lstinline{add :: Int -> Int -> Int} spezifizieren, die genau die Addition durchführt:
        \begin{lstlisting}
add :: Int -> Int -> Int
add n 0 = n
add n m = 1 + add n (m-1)
        \end{lstlisting}
    \end{frame}

	\begin{frame}[fragile] \frametitle{Ein weiteres Beispiel}
		\scriptsize
		Um die Analogie zwischen Funktionen in Mathe und Funktionen in Haskell noch einmal zu verdeutlichen, betrachten wir die Funktion
		\begin{equation*}
			\begin{split}
				f \colon \N &\to \N \\
				f(x) &= x + 3
			\end{split}
		\end{equation*}
		Diese würde in Haskell wie folgt aussehen:
		\begin{lstlisting}[basicstyle=\ttfamily]
f :: Int -> Int
f x = x + 3
		\end{lstlisting}
		\pause
		Außerdem kennen wir bereits die Variante Funktionen auf ihren Argumenten zu definieren, z.B.
		\begin{equation*}
			\begin{split}
				g \colon \N &\to \N \\
				g(0) &= 1 \\
				g(x) &= x^2
			\end{split}
		\end{equation*}
		bzw. in Haskell
		\begin{lstlisting}[basicstyle=\ttfamily]
g :: Int -> Int
g 0 = 1
g x = x^2
		\end{lstlisting}
	\end{frame}

	\begin{frame}\frametitle{Das Prinzip der Rekursion}
		\footnotesize
		\begin{theorem}
			Sei $B$ eine Menge, $b \in B$ und $F \colon B \times \mathbb{N} \to B$ eine Abbildung. Dann liefert die Vorschrift
			\begin{subequations}
				\begin{align}
				f(0) &:= b \\
				f(n+1) &:= F(f(n),n) \quad \forall n \in \mathbb{N} \label{eq: rekursionsschritt}
				\end{align}
			\end{subequations}
			genau eine Abbildung $f : \mathbb{N} \to B$.
		\end{theorem}
		\pause
		
		\emph{Beweis.} \qquad vollständige Induktion:
		\begin{description}
			\item[(IA)] Für $n=0$ ist $f(0) = b$ eindeutig definiert.
			\item[(IS)] Angenommen $f(n)$ sei eindeutig definiert. Wegen (\ref{eq: rekursionsschritt}) ist dann auch $f(n+1)$ eindeutig definiert.
		\end{description}
		
		\bigskip
		
		Man kann dieses Prinzip der Rekursion auf weitere Mengen (unabhängig von den natürlichen Zahlen) erweitern ($\nearrow$ Formale Systeme).
	\end{frame}
	
	\begin{frame}\frametitle{Haskell installieren und compilieren}
		\textbf{Glasgow Haskell Compiler} (ghc(i)) :  \url{https://www.haskell.org/ghc/}
		
		\begin{itemize}
			\item \textbf{Terminal:} \texttt{ghci <modulname>}
			\item \textbf{Module laden: } \texttt{:load <modulname>} oder \texttt{:l}
			\item \textbf{Module neu laden: } \texttt{:reload} oder \texttt{:r}
			\item \textbf{Hilfe: } \texttt{:?} oder \texttt{:help}
			\item \textbf{Interpreter verlassen: } \texttt{:quit} oder \texttt{:q}
			
			\bigskip
			
	 		\item \texttt{:type <exp>} --- Typ des Ausdrucks \texttt{<exp>} bestimmen
			\item \texttt{:info <fkt>} --- kurze Dokumentation für \texttt{<fkt>}
			\item \texttt{:browse} --- alle geladenen Funktionen anzeigen
			
			\bigskip
			
			\item einzeilige Kommentare mit \texttt{--}
			\item mehrzeilige Kommentare mit \texttt{\{- ... -\}}
		\end{itemize}
	\end{frame}
	%
	\begin{frame}[fragile] \frametitle{Grundlegende Strukturen -- Pattern Matching}
		\footnotesize
		Mit Pattern Matching kann man prüfen, ob Funktionsargumente eine bestimmte Form aufweisen.
		Zum Beispiel:
		\begin{itemize}
			\item Der Aufruf \texttt{func 3 0} passt auf das Pattern \texttt{func x 0}, aber nicht auf das Pattern \texttt{func 0 x}.
		\end{itemize}
		Damit kann man verschiedene Fälle in einfacher Form nacheinander abgreifen, z.B. Basis- und Rekursionsfall. Vergleiche dazu auch das Beispiel mit der \texttt{add}-Funktion:
		\begin{itemize}
			\item Der Aufruf \texttt{add 5 0} matched mit Zeile 2, also berechnen wir \texttt{add 5 0 = 5}.
			\item Der Aufruf \texttt{add 5 1} matched nicht auf Zeile 2, also probieren wir Zeile 3. Das matched mit \texttt{n = 5} und \texttt{m = 1} und wir berechnen
			\begin{lstlisting}[frame=none, numbers=none]
add 5 1 = 1 + add 5 0
        = 1 + 5
        = 6
			\end{lstlisting}
		\end{itemize}
		Beachte, dass dabei von oben nach unten getestet wird!
	\end{frame}
	\begin{frame}[fragile] \frametitle{Grundlegende Strukturen -- Conditionals}
		\footnotesize
		Um Bedingungen zu testen, gibt es die Möglichkeit auf \texttt{if}-\texttt{then}-\texttt{else} zu verzichten und sogenannte \textbf{guards} mit \textbf{pipes }zu verwenden. Das sieht dann wieder so aus, wie eine geschweifte Klammer in mathematischen Fallunterscheidungen. 
		\begin{equation*}
			h(x) = \begin{cases}
			x^2 & \text{für } x < 0 \\
			0.5*x & \text{für } \onslide*<1>{x \ge 0} \onslide*<2>{\cancel{x \ge 0} \text{ \textbf{sonst}}}
			\end{cases} 
		\end{equation*}
		\begin{onlyenv}<1>
			\begin{lstlisting}
h :: Int -> Int
h x 
	| x < 0  = x^2 
	| x <= 0 = 0.5 * x
			\end{lstlisting}
		\end{onlyenv}
		\begin{onlyenv}<2>
			\begin{lstlisting}
h :: Int -> Int
h x 
	| x < 0     = x^2 
	| otherwise = 0.5 * x
			\end{lstlisting}
		\end{onlyenv}
		\pause
		Wie auch in Mathe sollte man bei gegensätzlichen Bedingungen ein \enquote{sonst} bzw. \texttt{otherwise} verwenden.
	\end{frame}
	%
	\begin{frame}[fragile] \frametitle{Haskell \& Listen}
		\begin{description}
			\item[Listen] Wenn \texttt{a} ein Typ ist, dann bezeichnet \texttt{[a]} den Typ ``Liste mit Elementen vom Typ \texttt{a}'', insbesondere
			haben alle Elemente einer Liste den gleichen Typ
			\pause
			\item[cons-Operator `` \texttt{:} '']  Trennung von head und tail einer Liste \\
			\lstinline{[ x1 , x2 , x3 , x4 , x5] = x1 : [x2 , x3 , x4 , x5]}
			\bigskip \pause
			\item[Verkettungsoperator `` \texttt{++} ''] Verkettung zweier Listen gleichen Typs \\
			\lstinline{[x1 , x2 ] ++ [x3 , x4 , x5] = [x1 , x2 , x3 , x4 , x5]}
		\end{description}
	\end{frame}

	\begin{frame}[fragile] \frametitle{Zeichen \& Zeichenketten}
		\textbf{Zeichen}
		\begin{itemize}
			\item Datentyp \texttt{Char}
			\item Eingabe in einfachen Anführungszeichen
			\item z.B. \texttt{'a'}, \texttt{'e'}, \texttt{'3'}
		\end{itemize}
	
		\textbf{Zeichenketten}
		\begin{itemize}
			\item Datentyp \texttt{String = [Char]}
			\item Eingabe in doppelten Anführungszeichen
			\item z.B. \texttt{"hallo"}, \texttt{"welt"}
			\item Konkatenation von Zeichenketten: 
			\begin{lstlisting}[basicstyle=\ttfamily, numbers=none, frame=none, showstringspaces=false]
"hallo " ++ "welt" = "hallo welt"
			\end{lstlisting}
		\end{itemize}
	\end{frame}


\section{Übungsblatt 1}

\begin{frame}\frametitle{Aufgabe 2 -- Fakultätsfunktion}
	Fakultät
	\begin{equation*}
		n! = \prod_{i=1}^n i
	\end{equation*}
	
	\medskip
	\pause
	
	Daraus bilden wir nun eine Rekursionsvorschrift:
	\begin{equation*}
		n! = \boldsymbol{n} * \prod_{i=1}^{\boldsymbol{n-1}} i= n * (n-1)!
	\end{equation*}
	
	\medskip
	\pause
	
	Um die Rekursion vollständig zu definieren, benötigen wir einen (oder i.a. mehrere) \emph{Basisfall}. Wann können wir also die Rekursion der Fakultät abbrechen?
	\begin{equation*}
	0 ! = 1 \qquad 1 ! = 1 \qquad 2! = 2 \qquad \cdots
	\end{equation*}
	$\Rightarrow$ Welcher Basisfall ist sinnvoll? \qquad $0! = 1$
\end{frame}

\begin{frame}[fragile] \frametitle{Aufgabe 2 -- Lösung}
	\lstinputlisting{tut01-02.hs}
	
	\scriptsize
	\textbf{Hinweis:} In der Musterlösung werden noch \texttt{undefined}-Fälle angegeben. Das ist für uns erst einmal optional, aber natürlich schöner.
\end{frame}

\begin{frame}\frametitle{Aufgabe 3 -- Fibonacci-Zahlen}
	\scriptsize
	\begin{equation*}
	f_n := \begin{cases}
	\enskip 1 & \text{falls } n = 0 \\
	\enskip 1 & \text{falls } n = 1 \\
	\enskip f_{n-1} + f_{n-2} & \text{sonst}
	\end{cases}
	\end{equation*}
	
	\medskip
	\pause
	
	$\Rightarrow$ Rekursionsvorschrift schon gegeben. \\
	
	\bigskip 
	\begin{minipage}{\dimexpr0.5\linewidth-\fboxrule-\fboxsep}
		\textbf{Verfahren ohne Rekursion.} \\
		
		\begin{tikzpicture}[]%
		\pgfmathsetseed{1237}%
		\node (n1) [draw=cdindigo, fill=cdindigo!10,thick,align=justify, , inner ysep=2mm, inner xsep=2mm, rectangle] at (0,0) {$f_{i-2}$};%
		\node (n2) [draw=cdindigo, fill=cdindigo!10,thick,align=justify, , inner ysep=2mm, inner xsep=2mm, rectangle] at (1.5,0) {$f_{i-1}$};%
		\node (n3) [draw=cdindigo, fill=cdindigo!10,thick,align=justify, , inner ysep=2mm, inner xsep=2mm, rectangle] at (3,0) {$n$};%
		\node (n4) [draw=cdindigo, fill=cdindigo!10,thick,align=justify, , inner ysep=2mm, inner xsep=2mm, rectangle] at (0,-1) {$f_{i-1}$};%
		\node (n5) [draw=cdindigo, fill=cdindigo!10,thick,align=justify, , inner ysep=2mm, inner xsep=2mm, rectangle] at (1.5,-1) {$f_{i-2} + f_{i-1}$};%
		\node (n6) [draw=cdindigo, fill=cdindigo!10,thick,align=justify, , inner ysep=2mm, inner xsep=2mm, rectangle] at (3,-1) {$n-1$};%
		\path[->] (n2) edge [bend left=0] (n4);
		\end{tikzpicture}%
	\end{minipage}
	\pause
	\begin{minipage}{\dimexpr0.5\linewidth-\fboxrule-\fboxsep}
		\textbf{Explizite Formel.}
		\begin{equation*}
			f_n = \frac{\Phi^n - \left(-\frac{1}{\Phi} \right)^n}{\sqrt{5}}
		\end{equation*}
		mit $\Phi = \frac{1 + \sqrt{5}}{2}$
	\end{minipage}
\end{frame}

\begin{frame}[fragile] \frametitle{Aufgabe 3 -- Lösung}
	\lstinputlisting{tut01-03.hs}
\end{frame}

\end{document}