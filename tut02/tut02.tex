\documentclass{beamer}
\usepackage{../tut-slides}
\usepackage{../mathoperatorsAuD}

\usepackage{csquotes}
\usepackage{cancel}

\usepackage{amsmath,amssymb}
\usepackage{enumerate}
\usepackage[inline]{enumitem} 		%customize label
\newcommand{\labelitemi}{\raisebox{1pt}{\scalebox{.9}{$\blacktriangleright$}}}
\newcommand{\labelitemii}{$\vartriangleright$}
\newcommand{\labelitemiii}{--}

\usepackage{booktabs}
\usepackage{tabularx}
\usepackage{tabu}
\newcommand*\head{\rowfont{\bfseries}}
\newcommand*{\tw}{\rowfont{\ttfamily}}

\renewcommand{\tabularxcolumn}[1]{>{\hspace{0pt}}m{#1}}

\usepackage{listings}
\lstset{ 
	basicstyle=\footnotesize\ttfamily,        % the size of the fonts that are used for the code
	breakatwhitespace=false,         % sets if automatic breaks should only happen at whitespace
	breaklines=true,                 % sets automatic line breaking
	commentstyle=\itshape,    	     % comment style
	escapeinside={\%*}{*)},          % if you want to add LaTeX within your code
	extendedchars=true,              % lets you use non-ASCII characters; for 8-bits encodings only, does not work with UTF-8
	firstnumber=1,                % start line enumeration with line 1000
	frame=single,
	keywordstyle=\bfseries,       % keyword style
	morekeywords={}, 
	language=Haskell,                 % the language of the code
	numbers=left,                    % where to put the line-numbers; possible: (none, left, right)
	numbersep=5pt,                   % how far the line-numbers are from the code
	numberstyle=\tiny\color{cdgray!50}, % the style that is used for the line-numbers
	rulecolor=\color{cddarkblue}, 
	tabsize=2,	                   % sets default tabsize to 2 spaces
}

\begin{document}	
	\title{Programmierung}
	\subtitle{Übung 2: Listen, Zeichenketten \& Bäume}
	\author{Eric Kunze}
	\email{eric.kunze@mailbox.tu-dresden.de}
%	\city{TU Dresden}
	\date{}
%	\institute{Lehrstuhl für Grundlagen der Programmierung}
%	\titlegraphic{\includegraphics[width=2cm]{../TUD-white.pdf}}
	
	\maketitle
	
	

	\begin{frame} \frametitle{Hinweise}
		
		\begin{itemize}
			\item Übungsaufgaben können \textit{freiwillig} abgegeben werden
			\item keine Bonuspunkte
			\item Abgaben in möglichst einer Datei und (solange es geht) nur Quelltexte
			
			\bigskip
			
			\item Hinweise meinerseits nur noch auf meiner Website \url{oakoneric.github.io}
			\item Fragen und Anmerkungen jederzeit
			\item einfache Fragen (Orga oder kleine Inhalte) gern schnell und unkompliziert über Telegram
		\end{itemize}
		\centering \fbox{\textbf{Probleme aus der vergangenen Woche?}}
	\end{frame}


%%%%%%%%%%%%%%%%%%%%%%%%%%%%%%%%%%%%%%%%%%%%%%%%%%%%%%%%%%%%%%%%%%%%%%%%%%%%%

\section{Listen \& Zeichenketten in Haskell}
	%
	\begin{frame}[fragile] \frametitle{Listen}
		\begin{description}
			\item[Listen] Wenn \texttt{a} ein Typ ist, dann bezeichnet \texttt{[a]} den Typ ``Liste mit Elementen vom Typ \texttt{a}'', insbesondere
			haben alle Elemente einer Liste den gleichen Typ
			\pause
			\item[cons-Operator `` \texttt{:} '']  Trennung von \textit{head} und \textit{tail} einer Liste \\
			\lstinline{[ x1 , x2 , x3 , x4 , x5] = x1 : [x2 , x3 , x4 , x5]}
			\bigskip \pause
			\item[Verkettungsoperator `` \texttt{++} ''] Verkettung zweier Listen gleichen Typs \\
			\lstinline{[x1 , x2 ] ++ [x3 , x4 , x5] = [x1 , x2 , x3 , x4 , x5]}
		\end{description}
	\end{frame}

	\begin{frame}[fragile] \frametitle{Zeichen \& Zeichenketten}
		\textbf{Zeichen}
		\begin{itemize}
			\item Datentyp \texttt{Char}
			\item Eingabe in einfachen Anführungszeichen
			\item z.B. \texttt{'a'}, \texttt{'e'}, \texttt{'3'}
		\end{itemize}
		\pause
		\textbf{Zeichenketten}
		\begin{itemize}
			\item Datentyp \texttt{String = [Char]}
			\item Eingabe in doppelten Anführungszeichen
			\item z.B. \texttt{"hallo"}, \texttt{"welt"}
			\item Konkatenation von Zeichenketten: 
			\begin{lstlisting}[basicstyle=\ttfamily, numbers=none, frame=none, showstringspaces=false]
"hallo " ++ "welt" = "hallo welt"
			\end{lstlisting}
		\end{itemize}
	\end{frame}

	\begin{frame} \frametitle{Übung: Datentypen}
		Bestimme den Datentyp:
		\begin{itemize}
			\item \texttt{"prog"}
			\item \texttt{5}
			\item \texttt{'3'}
			\item \texttt{mod} (gemeint ist die Modulo-Funktion)
			\item \texttt{[2,4,6,8]}
			\item \texttt{[[" \!\!\! e", "r", "\!\!\! i", "c"], ["k", "\!\!\! u", "n", " \!\!\! z", "\!\!\! e"]]}
		\end{itemize}
	\end{frame}


\section{Übungsblatt 2 \\ \textit{\normalsize Aufgabe 1}}

\begin{frame}[t, fragile] \frametitle{Aufgabe 1 -- Teil (a)}
	\textbf{Multiplikation einer Liste}
	
	\texttt{prod :: [Int] -> Int}
	
	\pause \bigskip
	
	\begin{lstlisting}
-- (a) Produkt der Listenelemente
prod :: [Int] -> Int
prod []     = 1
prod (x:xs) = x * prod xs
	\end{lstlisting}
\end{frame}

\begin{frame}[t, fragile] \frametitle{Aufgabe 1 -- Teil (b)}
	\textbf{Umkehrung einer Liste}
	
	\texttt{rev :: [Int] -> [Int]}
	
	\pause \bigskip
	
	\begin{lstlisting}
rev :: [Int] -> [Int]
rev []     = []
rev (x:xs) = rev xs ++ [x]
	\end{lstlisting}
\end{frame}

\begin{frame}[t, fragile] \frametitle{Aufgabe 1 -- Teil (c)}
	\textbf{Elemente einer Liste löschen}
	
	\texttt{excl :: Int -> [Int] -> [Int]}
	
	\pause \bigskip
	
	\begin{lstlisting}
excl :: Int -> [Int] -> [Int]
excl _ [] = []
excl n (x:xs)
	| x /= n    = x : excl n xs
	| otherwise = excl n xs
	\end{lstlisting}
\end{frame}

\begin{frame}[t, fragile] \frametitle{Aufgabe 1 -- Teil (d)}
	\textbf{Sortierung einer Liste prüfen}
	
	\texttt{isOrd :: [Int] -> Bool}
	
	\pause \bigskip
	\begin{lstlisting}
isOrd :: [Int] -> Bool
isOrd []  = True
isOrd [x] = True
isOrd (x:y:xs)
	| x <= y    = isOrd (y:xs)
	| otherwise = False
	\end{lstlisting}
	\pause
	\begin{lstlisting}
isOrd' :: [Int] -> Bool
isOrd' []       = True
isOrd' [x]      = True
isOrd' (x:y:xs) = x <= y && isOrd' (y:xs)
	\end{lstlisting}
\end{frame}

\begin{frame}[t, fragile] \frametitle{Aufgabe 1 -- Teil (e)}
	\textbf{sortiertes Zusammenfügen zweier (sortierten) Listen}
	
	\texttt{merge :: [Int] -> [Int] -> [Int]}
	
	\pause \bigskip
	\begin{lstlisting}
merge :: [Int] -> [Int] -> [Int]
merge [] ys = ys
merge xs [] = xs
merge (x:xs) (y:ys)
	| x < y     = x : merge xs (y:ys)
	| otherwise = y : merge (x:xs) ys
	\end{lstlisting}
	\pause
	\begin{lstlisting}
merge' :: [Int] -> [Int] -> [Int]
merge' [] ys = ys
merge' xs [] = xs
merge' xxs@(x:xs) yys@(y:ys)
	| x < y     = x : merge' xs yys
	| otherwise = y : merge' xxs ys
	\end{lstlisting}
\end{frame}

\begin{frame}[t, fragile] \frametitle{Aufgabe 1 -- Teil (f)}
	\textbf{(unendliche) Liste der Fibonacci-Zahlen}
	
	\texttt{fibs :: [Int]}
	
	\pause \bigskip
	\begin{lstlisting}
fibs :: [Int]
fibs = fibs' 0 1
	where fibs' n m = n : fibs' m (n+m)
	\end{lstlisting}
	\pause
	\begin{lstlisting}
fib :: Int -> Int
fib 0 = 1
fib 1 = 1
fib n = fib (n-1) + fib (n-2)
	
fibs :: [Int]
fibs = fibAppend 0
	where fibAppend x = fib x : fibAppend (x+1)
	\end{lstlisting}
\end{frame}

%%%%%%%%%%%%%%%%%%%%%%%%%%%%%%%%%%%%%%%%%%%%%%%%%%%%%%%%%%%%%%%%%%%%%%%%%%%%%%%%%%%%

\section{Übungsblatt 2 \\ \textit{\normalsize Aufgabe 2}}

\begin{frame}[t, fragile] \frametitle{Aufgabe 2 -- Teil (a)}
	\textbf{Liste von Wörtern aneinanderfügen}
	
	\texttt{join :: [String] -> String}
	
	\pause \bigskip
	
	\begin{lstlisting}
join :: [String] -> String
join [] = ""
join [x] = x
join (x:xs) = x  ++ ' ' : join xs
	\end{lstlisting}
\end{frame}

\begin{frame}[t, fragile] \frametitle{Aufgabe 2 -- Teil (a)}
	\textbf{Trennung eines Strings in seine Wörter}
	
	\texttt{unjoin :: String -> [String]}
	
	\bigskip
	
	\begin{onlyenv}<2>
		\begin{lstlisting}
unjoin :: String -> [String]
unjoin s = f [] s
	where
		f save [] = [save]
		f save (c:cs)
			| c == ' ' = save : f [] cs
			| otherwise = f (save ++ [c]) cs
		\end{lstlisting}
	\end{onlyenv}
	
	\begin{onlyenv}<3>
		\begin{lstlisting}
unjoin'' :: String -> [String]
unjoin''     [] = []
unjoin''    [c] = if c == ' ' then [[], []] else [[c]]
unjoin'' (c:cs)
	| c == ' '  = [] : unjoin'' cs
	| otherwise = let (s:ss) = unjoin'' cs
			      in ((c:s):ss)
		\end{lstlisting}
	\end{onlyenv}
\end{frame}

%%%%%%%%%%%%%%%%%%%%%%%%%%%%%%%%%%%%%%%%%%%%%%%%%%%%%%%%%%%%%%%%%%%%%%%%%%%%%%%%%%%%

\section{Übungsblatt 2 \\ \textit{\normalsize Aufgabe 3}}

\begin{frame}[fragile] \frametitle{Algebraische Datentypen}
	\begin{itemize}
		\item Ziel: problemspezifische Datenkonstruktoren
		\item z.B. in \textit{C}: Aufzählungstypen
		\item funktionale Programmierung: algebraische Datentypen
	\end{itemize}

	\textbf{Aufbau:}
	\begin{lstlisting}
data Typename 
	= Con1 t11 ... t1k1
	| Con2 t21 ... t2k2
	| ...
	| Conr tr1 ... trkr
	\end{lstlisting}
	\begin{itemize}
		\item \texttt{Typename} ist ein Name (Großbuchstabe)
		\item \texttt{Con1, ... Conr} sind Datenkonstruktoren (Großbuchstabe)
		\item \texttt{tij} sind Typnamen (Großbuchstaben)
	\end{itemize}
\end{frame}

\begin{frame}[fragile] \frametitle{Algebraische Datentypen -- Beispiele}
	
	\begin{lstlisting}
data Typename 
	= Con1 t11 ... t1k1
	| Con2 t21 ... t2k2
	| ...
	| Conr tr1 ... trkr
	\end{lstlisting}
	
	\begin{lstlisting}
data Season = Spring | Summer | Autumn | Winter
	\end{lstlisting}
	
	\begin{lstlisting}
goSkiing :: Season -> Bool
goSkiing Winter = True
goSkiing _      = False
	\end{lstlisting}
	
	\begin{lstlisting}
data TriBool = TriTrue | TriMaybe | TriFalse
	\end{lstlisting}
\end{frame}

\begin{frame}[t, fragile] \frametitle{Aufgabe 3}
	\begin{lstlisting}
data BinTree = Branch Int BinTree BinTree | Nil
	\end{lstlisting}
	\pause
	\begin{lstlisting}[firstnumber=2, basicstyle=\ttfamily\scriptsize]
tree1 :: BinTree -- Suchbaum  
tree1 = Branch 5 
		(   Branch 3 
				(Branch 2 Nil Nil) 
				(Branch 4 Nil Nil)
		)(
				Branch 8 
					(   Branch 7 
							(Branch 6 Nil Nil) 
							(Nil)
					)
					(   Branch 10 
							(Nil)
							(Branch 13 Nil Nil)
					)
		)
	\end{lstlisting}
\end{frame}

\begin{frame}[t, fragile] \frametitle{Aufgabe 3 -- Teil (a)}
	\textbf{Einfügen von Schlüsseln in einen Binärbaum}
	
	\begin{ttfamily}
		data BinTree = Branch Int BinTree BinTree | Nil \\
		insert :: BinTree -> [Int] -> BinTree
	\end{ttfamily}
	
	\bigskip \pause
	
	\begin{lstlisting}
insert :: BinTree -> [Int] -> BinTree
insert t     [] = t
insert t (x:xs) = insert t' xs
	where
		t' = insertSingle t x
		insertSingle Nil            x = Branch x Nil Nil
		insertSingle (Branch y l r) x
			| x < y     = Branch y (insertSingle l x) r
			| otherwise = Branch y l                  (insertSingle r x)
	\end{lstlisting}
\end{frame}

\begin{frame}[t, fragile] \frametitle{Aufgabe 3 -- Teil (b)}
	\textbf{Test auf Baum-Gleichheit}
	
	\begin{ttfamily}
		data BinTree = Branch Int BinTree BinTree | Nil \\
		equal :: \only<2>{BinTree -> BinTree -> Bool}
	\end{ttfamily}
	
	\bigskip \pause
	
	\begin{lstlisting}
equal :: BinTree -> BinTree -> Bool
equal Nil              Nil              = True
equal Nil              (Branch y l2 r2) = False
equal (Branch x l1 r1) Nil              = False
equal (Branch x l1 r1) (Branch y l2 r2)
	= (x == y) && (equal l1 l2) && (equal r1 r2)
	\end{lstlisting}
\end{frame}

\begin{frame} \frametitle{Ende}
	\centering
	\textbf{Fragen?}
\end{frame}

\end{document}