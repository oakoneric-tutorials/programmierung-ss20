\documentclass{beamer}
\usepackage{../tut-slides}
\usepackage{../mathoperatorsAuD}

\usepackage{csquotes}
\usepackage{cancel}

\usepackage{amsmath,amssymb}


\usepackage{booktabs}
\usepackage{tabularx}
\usepackage{tabu}
\newcommand*\head{\rowfont{\bfseries}}
\newcommand*{\tw}{\rowfont{\ttfamily}}

\renewcommand{\tabularxcolumn}[1]{>{\hspace{0pt}}m{#1}}

\usepackage{listings}
\lstset{ 
	basicstyle=\footnotesize\ttfamily,        % the size of the fonts that are used for the code
	breakatwhitespace=false,         % sets if automatic breaks should only happen at whitespace
	breaklines=true,                 % sets automatic line breaking
	commentstyle=\itshape,    	     % comment style
	escapeinside={\%*}{*)},          % if you want to add LaTeX within your code
	extendedchars=true,              % lets you use non-ASCII characters; for 8-bits encodings only, does not work with UTF-8
	firstnumber=1,                % start line enumeration with line 1000
	frame=single,
	keywordstyle=\bfseries,       % keyword style
	morekeywords={}, 
	language=Haskell,                 % the language of the code
	numbers=left,                    % where to put the line-numbers; possible: (none, left, right)
	numbersep=5pt,                   % how far the line-numbers are from the code
	numberstyle=\tiny\color{cdgray!50}, % the style that is used for the line-numbers
	rulecolor=\color{cddarkblue}, 
	tabsize=2,	                   % sets default tabsize to 2 spaces
}

\usepackage{textgreek}

\DeclareMathOperator{\GV}{GV}
\DeclareMathOperator{\FV}{FV}

\renewcommand{\emph}[1]{\textbf{#1}}
\newcommand{\coloremph}[1]{\textcolor{cdpurple}{#1}}
\newcommand{\col}[1]{\textcolor{cdpurple}{\boldsymbol{#1}}}
\newcommand{\coll}[1]{\textcolor{cddarkgreen}{\boldsymbol{#1}}}
\newcommand{\colll}[1]{\textcolor{cdorange}{\boldsymbol{#1}}}
%\newcommand{\step}[2][]{\ensuremath{\overset{{#1} (\text{#2})}{=}}}
%\newcommand*{\astep}[2][]{\ensuremath{\overset{{#1} (\text{#2})}&{=}}}

\newcommand{\num}[1]{\ensuremath{\langle #1 \rangle}}

\begin{document}	
	\title{Programmierung}
	\subtitle{Übung 6: \textlambda-Kalkül (Teil 2)}
	\author{Eric Kunze}
	\email{eric.kunze@mailbox.tu-dresden.de}
%	\city{TU Dresden}
	\date{}
%	\institute{Lehrstuhl für Grundlagen der Programmierung}
%	\titlegraphic{\includegraphics[width=2cm]{../TUD-white.pdf}}
	
	\maketitle
	

%%%%%%%%%%%%%%%%%%%%%%%%%%%%%%%%%%%%%%%%%%%%%%%%%%%%%%%%%%%%%%%%%%%%%%%%%%%%%

\begin{frame}[fragile] \frametitle{Inhalt}
	\begin{enumerate}
		\item Funktionale Programmierung
		\begin{enumerate}
			\item Einführung in Haskell: Listen
			\item Algebraische Datentypen
			\item Funktionen höherer Ordnung
			\item Typpolymorphie \& Unifikation
			\item Beweis von Programmeigenschaften
			\item \textbf{\textlambda--Kalkül}
		\end{enumerate}
		\item Logikprogrammierung
		\item Implementierung einer imperativen Programmiersprache
		\item Verifikation von Programmeigenschaften
		\item H${}_\text{0}$ -- ein einfacher Kern von Haskell
	\end{enumerate}
\end{frame}


\section{Der \textlambda--Kalkül}

\begin{frame}\frametitle{Der $\lambda$-Kalkül}
	\begin{tabbing}
		\textbf{Atome.} $\qquad \quad \enskip$ \= $x,y$ \\
		\textbf{Abstraktion.} \> $(\lambda x.t)$ \hspace{1cm} \textcolor{cdgray}{($f(x) = t$, anonyme Funktion)} \\
		\textbf{Applikation.} \> $(t_1 \enskip  t_2)$
	\end{tabbing}
	
	\textbf{Verabredungen:}
	\begin{itemize}
		\item Applikation ist \textit{linksassoziativ}: $((t_1 \, t_2) \, t_3) = t_1 \, t_2 \, t_3 \enskip \text{für alle } t_1, t_2, t_3 \in \lambda(\Sigma)$
		\item mehrfache Abstraktion: $(\lambda x_1 . (\lambda x_2 . (\lambda x_3 . t)))) = (\lambda x_1 x_2 x_3 . t) \enskip \text{für alle } t \in \lambda(\Sigma)$
		\item Applikation vor Abstraktion:
		\begin{align*}
			(\lambda x. xy) 
			&= (\lambda x . (xy)) \\
			&\neq ((\lambda x.x) y)
		\end{align*}
	\end{itemize}
\end{frame}


\begin{frame} \frametitle{$\alpha$-Konversion \& $\beta$-Reduktion}
	\begin{block}{\textbeta--Reduktion}
		Seien $s,t \in \lambda(\Sigma)$ gültige $\lambda$-Terme und es gilt $\GV(t) \cap \FV(s) = \emptyset$. 
		\begin{equation*}
			(\lambda x . t) \ s \quad \longrightarrow_\beta \quad t[x / s]
		\end{equation*}
	\end{block}
	
	\begin{block}{\textalpha --Konversion}
		Sei $t \in \lambda(\Sigma)$ und $z \notin \GV(t) \cup \FV(t)$.
		\begin{equation*}
		(\lambda x . t) \quad \longrightarrow_\alpha \quad \lambda z . t[x / z]
		\end{equation*}
	\end{block}
\end{frame}

\begin{frame} \frametitle{Church-Numerals}
	Dadurch, dass wir im Folgenden keine Symbole mehr zulassen (d.h. $\Sigma = \emptyset$), benötigen wir eine alternative Charakterisierung dieser.
	Zuerst beschäftigen uns die natürlichen Zahlen.
	
	\textbf{Darstellung der natürlichen Zahlen:} \textit{Church-Numerals} 
	\begin{align*}
		\num{0} &= (\lambda xy \ . \ y) \\ 
		\num{1} &= (\lambda xy \ . \ xy) \\
		\num{2} &= (\lambda xy \ . \ x(xy)) \\
		&\vdots \\
		\num{n} &= (\lambda xy \ . \ \underbrace{x (x \dots (x}_n y ) \dots ))
	\end{align*}
\end{frame}


\begin{frame} \frametitle{Fixpunktkombinator und Rekursion}
	\begin{itemize}
		\item Ein $t \in \Sigma(\lambda)$ heißt \textbf{geschlossener Term}, falls $FV(t) = \emptyset$. Ein geschlossender Term heißt auch \textbf{Kombinator}.
		\item \textbf{Fixpunktkombinator.} $\num{Y} = \Bigl( \lambda z . \ \bigl( \lambda u . z (uu) \bigr) \ \bigl( \lambda u. z (uu) \bigr) \Bigr) \in \lambda(\emptyset)$
		\item Der Fixpunktkombinator ermöglicht Rekursion.
		\item weitere definierte $\lambda$-Terme (siehe Skript S. 198f.):
		\begin{align*}
			\num{true} &= (\lambda xy. x) & \num{false} &= (\lambda xy. y) \\
			\num{succ} &= (\lambda z . (\lambda xy . x (zxy))) &  \num{pred}\num{0} &\Rightarrow^\ast \num{0} \\
			\num{succ}\num{n} &\Rightarrow^\ast \num{n+1} & 
			\num{pred}\num{n} &\Rightarrow^\ast \num{n-1} \\
		\end{align*}
		\begin{equation*}
			\num{ite} \ s \ s_1 \ s_2 \Rightarrow^\ast \begin{cases}
			s_1 & \text{wenn } s \Rightarrow^\ast \num{true} \\
			s_2 & \text{wenn } s \Rightarrow^\ast \num{false}
			\end{cases}
		\end{equation*}
	\end{itemize}
\end{frame}


\section{Übungsblatt 7 \\ \textit{\normalsize Aufgabe 1}}


\begin{frame} \frametitle{Aufgabe 1 -- Teil (a)}
	\begin{align*}
		&(\lambda f \underbrace{x . f \ f \ x}_{\GV = \menge{x}}) \ (\underbrace{\lambda y . x}_{\FV = \menge{x}}) \ z	\\
		\Rightarrow_\alpha \quad
		& (\lambda f \underbrace{x_1 . f \ f \ x_1}_{\GV = \menge{x_1}}) \ (\underbrace{\lambda y . x}_{\FV = \menge{x}}) \ z
		\\
		\Rightarrow_\beta \quad
		& (\lambda x_1 . (\lambda y . \underbrace{x}_{\GV = \emptyset}) \ (\underbrace{\lambda y . x}_{\FV = \menge{x}}) \ x_1) \ z
		\\
		\Rightarrow_\beta \quad
		& (\lambda x_1 . \underbrace{x x_1}_{\GV = \emptyset}) \ \underbrace{z}_{\FV = \menge{z}}
		\\
		\Rightarrow_\beta \quad
		& xz
		\\
	\end{align*}
\end{frame}

\begin{frame} \frametitle{Aufgabe 1 -- Teil (b)}
	\tiny
	\begin{equation*}
		\num{F} = \Biggl( \lambda f xyz \ . \ \num{ite} \ \Bigl(\num{iszero} \ (\num{sub} x y )\Bigr) \ \Bigl( \num{add} y z \Bigr) \ \Bigl( \num{succ} \ ( f \ (\num{pred} x) \ (\num{succ} y) \ ( \num{mult} \num{2} z )) \Bigr) \Biggr)
	\end{equation*}
	
	\normalsize
	
	\textbf{Nebenrechnung:} Zeige die Wirkung des Fixpunktkombinators.
	\begin{align*}
		\num{Y} \num{F}
		= \enskip &\Bigl( \lambda z . \ \bigl( \lambda u . z (uu) \bigr) \ \bigl( \lambda u. z (uu) \bigr) \Bigr) \ \num{F} \\
		\Rightarrow^{\beta} \quad &\Bigl( \lambda u . \num{F} (uu) \Bigr) \ \Bigl( \lambda u. \num{F} (uu) \Bigr) \qquad =: \num{Y_F} \\
		\Rightarrow^{\beta} \quad &\num{F} \num{Y_F}
	\end{align*}
\end{frame}
%
\begin{frame} \frametitle{Aufgabe 1 -- Teil (b)}
	\footnotesize
	\begin{align*}
		\num{Y} \num{F} \num{6} \num{5} \num{3} &\Rightarrow^\ast \num{F} \num{Y_F} \num{6} \num{5} \num{3} \\
		&\Rightarrow^\ast \num{ite} \ 
		(\underbrace{\num{iszero} \ (\num{sub}\num{6}\num{5})}_{\Rightarrow^\ast \num{false}}) \ 
		( \dots ) \\
		& \phantom{\Rightarrow^\ast \num{ite}} \ (\num{succ} (\num{Y_F}(\underbrace{\num{pred}\num{6}}_{\Rightarrow^\ast \num{5}})(\underbrace{\num{succ}\num{5}}_{\Rightarrow^\ast \num{6}})(\underbrace{\num{mult}\num{2}\num{3}}_{\Rightarrow^\ast \num{6}}))) \\
		&\Rightarrow^\ast \num{succ} \  ( \ \num{Y_F}\num{5}\num{6}\num{6} \ ) \\
		&\Rightarrow^\ast \num{succ} \ ( \ \num{F} \num{Y_F} \num{5}\num{6}\num{6} \ ) \\
		&\Rightarrow^\ast \num{succ} \ ( \ \num{ite} \  (\underbrace{\num{iszero}(\num{sub}\num{5}\num{6})}_{\Rightarrow^\ast \num{true}}) \ (\underbrace{\num{add}\num{6}\num{6}}_{\Rightarrow^\ast \num{12}})  \ (\dots) \ ) \\
		&\Rightarrow^\ast \num{succ} \ \num{12} \\
		&\Rightarrow^\ast \num{13}
	\end{align*}
\end{frame}

\begin{frame} \frametitle{Aufgabe 1 -- Teil (c)}
	\small
	\begin{alignat*}{3}
		\num{G} = \enskip & \Biggl( \lambda g xy \ . \ \Bigl( &&\num{ite} \ && \bigl( \num{iszero} \ x \bigr) \\
		&&&&& \bigl( \num{mult} \ \num{2} \ (\num{succ} y) \ \bigr) \\
		&&&&& \Bigl( 
			\begin{alignedat}[t]{1}
			\num{ite} \ &(\num{iszero} \ y) \\
			& \bigl( \num{mult} \ \num{2} \ (\num{succ} x) \bigr) \\
			& \bigl( \num{add} \ \num{4} \enskip g \ (\num{pred} x \ \num{pred} y)  \ \bigr) \\				
		\end{alignedat} \\
		&&&&& \Bigr) \\
		&&& \Bigr) \\
		& \Biggr)
	\end{alignat*}
\end{frame}


\section{Übungsblatt 7 \\ \textit{\normalsize Aufgabe 2}}

\begin{frame} \frametitle{Aufgabe 2 -- Teil (a)}
	\small
	\begin{equation*}
		g \colon \mathbb{N} \times \mathbb{N} \to \mathbb{N} \quad \text{ mit } \quad g(x,y) := \begin{cases}
		x * x & \text{für } y=0 \\
		g(2*x,y-1) & \text{für } y \ge 1 
		\end{cases}
	\end{equation*}
	
	\medskip
	
	\begin{alignat*}{3}
		\num{G} = \enskip & \Biggl( \lambda g xy \ . \ \biggl( &&\num{ite} \ && \Bigl( \num{iszero} \ y \Bigr) \\
		&&&&& \Bigl( \num{mult} \ x \ x \ \Bigr) \\
		&&&&& \Bigl( g \ \bigl( \num{mult} \ \num{2} \ x \bigr) \ \bigl( \num{pred} \ y \bigr) \Bigr) \\
		&&& \biggr) \\
		& \Biggr)
	\end{alignat*}
\end{frame}

\begin{frame} \frametitle{Aufgabe 2 -- Teil (b)}
	\footnotesize
	\begin{equation*}
		\num{G} = \Biggl( \lambda \ g xy \ . \ \num{ite} \ \Bigl(\num{iszero} \ y \Bigr) \ \Bigl( \num{mult} \ x \ x \Bigr) \ \Bigl( g \ \bigl( \num{mult} \ \num{2} \ x \bigr) \ \bigl(\num{pred} \ y) \bigr) \Bigr) \Biggr)
	\end{equation*}
	
	\bigskip 
	\normalsize
	
	\textbf{Nebenrechnung:} Zeige die Wirkung des Fixpunktkombinators.
	\begin{align*}
		\num{Y} \num{G}
		= \enskip &\Bigl( \lambda z . \ \bigl( \lambda u . z (uu) \bigr) \ \bigl( \lambda u. z (uu) \bigr) \Bigr) \ \num{G} \\
		\Rightarrow^{\beta} \quad &\Bigl( \lambda u . \num{G} (uu) \Bigr) \ \Bigl( \lambda u. \num{G} (uu) \Bigr) \qquad =: \num{Y_G} \\
		\Rightarrow^{\beta} \quad &\num{G} \num{Y_G}
	\end{align*}
\end{frame}

\begin{frame} \frametitle{Aufgabe 2 -- Teil (b)}
	\scriptsize
	\begin{alignat*}{2}
	&\num{Y} \num{G} \num{1} \num{3} \\
	\Rightarrow^\ast &\num{G} \num{Y_G} \num{1} \num{3} \\
	\Rightarrow^\ast 
	&\num{ite} \ 
	\Bigl( \underbrace{\num{iszero} \ \num{3}}_{\Rightarrow^\ast \num{false}} \Bigr) \ \Bigl( \dots \Bigr) \ \Bigl( \num{Y_G} \ \bigl( \underbrace{\num{mult} \ \num{2} \ \num{1}}_{\Rightarrow^\ast \num{2}} \bigr) \ \bigl(\underbrace{\num{pred} \ \num{3}}_{\Rightarrow^\ast \num{2}} \bigr) 
	&&\Rightarrow^\ast 
	\num{G} \ \num{Y_G} \ \num{2} \ \num{2} \\
	%
	\Rightarrow^\ast 
	&\num{ite} \ 
	\Bigl( \underbrace{\num{iszero} \ \num{2}}_{\Rightarrow^\ast \num{false}} \Bigr) \ \Bigl( \dots \Bigr) \ \Bigl( \num{Y_G} \ \bigl( \underbrace{\num{mult} \ \num{2} \ \num{2}}_{\Rightarrow^\ast \num{4}} \bigr) \ \bigl(\underbrace{\num{pred} \ \num{2}}_{\Rightarrow^\ast \num{1}} \bigr) \Bigr)
	&&\Rightarrow^\ast 
	\num{G} \ \num{Y_G} \ \num{4} \ \num{1} \\
	%
	\Rightarrow^\ast 
	&\num{ite} \ 
	\Bigl( \underbrace{\num{iszero} \ \num{1}}_{\Rightarrow^\ast \num{false}} \Bigr) \ \Bigl( \dots \Bigr) \ \Bigl( \num{Y_G} \ \bigl( \underbrace{\num{mult} \ \num{2} \ \num{4}}_{\Rightarrow^\ast \num{8}} \bigr) \ \bigl(\underbrace{\num{pred} \ \num{1}}_{\Rightarrow^\ast \num{0}} \bigr) \Bigr)
	&&\Rightarrow^\ast
	\num{G} \ \num{Y_G} \ \num{8} \ \num{0} \\
	%
	\Rightarrow^\ast
	&\num{ite} \ 
	\Bigl( \underbrace{\num{iszero} \ \num{0}}_{\Rightarrow^\ast \num{true}} \Bigr) \ \Bigl( \underbrace{\num{mult} \ \num{8} \ \num{8}}_{\Rightarrow^\ast \num{64}} \Bigr) \ \Bigl( \dots \Bigr)
	%
	&&\Rightarrow^\ast \num{64}
	\end{alignat*}
\end{frame}	



\begin{frame} \frametitle{Ende}
	\centering
	\textbf{Fragen?}
\end{frame}


\section{Übungsblatt 7 \\ \textit{\normalsize Zusatzaufgabe 1}}

\begin{frame} \frametitle{Zusatzaufgabe 1}
	\scriptsize
	\begin{align*}
		\num{pow} \num{2} = \enskip &\Bigl( \lambda \col{n}fz \ . \ \col{n} \ ( \colll{\lambda gx . g (gx)}) \ f z \Bigr) \, \col{\Bigl( \lambda xy \ .  \ x(xy)) \Bigr)} \\
		%
		\Rightarrow^{\beta} \enskip &\Bigl( \lambda fz \ . \ \Bigl( \lambda \col{x}y \ .  \ \col{x}(\col{x}y)) \Bigr) \ \col{\Bigl( \lambda gx . g (gx) \Bigr)} \ f \ z \Bigr) \\
		%
		\Rightarrow^{\beta} \enskip &\Bigl( \lambda fz \ . \ \biggl( \lambda y \ . \  \Bigl( \lambda \col{g}x . \col{g} (\col{g}x) \Bigr) \, \col{\Bigl( (\lambda gx . g (gx) ) \ y \Bigr)} \biggr) \  f \ z \Bigr) \\
		%
		\Rightarrow^{\beta} \enskip &\Bigl( \lambda fz \ . \ \biggl( \lambda y \ . \  \Bigl( \coll{\lambda x} . \Bigl( (\lambda gx . g (gx) ) \ y \Bigr) \, \Bigl( \bigl( (\lambda gx . g (gx) ) \ y \bigr) \ x \Bigr) \, \Bigr) \,  \biggr) \  f \ z \Bigr) \\
		%
		\Rightarrow^{\beta} \enskip &\Bigl( \lambda fz \ . \ \biggl( \lambda y \coll{x} \ . \Bigl( (\lambda \col{g}x . \col{g} (\col{g}x) ) \ \col{y} \Bigr) \, \Bigl( \bigl( (\lambda \col{g}x . \col{g} (\col{g}x) ) \ \col{y} \bigr) \ x \Bigr) \,  \biggr) \  f \ z \Bigr) \\
		%
		\Rightarrow^{\beta} \enskip &\Bigl( \lambda fz \ . \ \biggl( \lambda y x \ . \Bigl( \lambda x . y (yx)  \Bigr) \, \Bigl( (\lambda \col{x} . y (y\col{x}) ) \ \col{x} \Bigr) \,  \biggr) \  f \ z \Bigr) \\
		%
		\Rightarrow^{\beta} \enskip &\Bigl( \lambda fz \ . \ \biggl( \lambda y x \ . \Bigl( \lambda \col{x} . y (y\col{x})  \Bigr) \, \col{\Bigl( y (yx) \Bigr)} \,  \biggr) \  f \ z \Bigr) \\
		%
		\Rightarrow^{\beta} \enskip &\Bigl( \lambda fz \ . \ \biggl( \lambda \col{y} x \ . \col{y} \ \Bigl(\col{y} \ \bigl( \col{y} \ (\col{y}x) \bigr)\Big) \  \biggr) \  \col{f} \ z \Bigr) \\
		%
		\Rightarrow^{\beta} \enskip &\Bigl( \lambda fz \ . \ \biggl( \lambda \col{x} \ . f \ \Bigl(f \ \bigl( f \ (f \col{x}) \bigr)\Big) \  \biggr) \ \col{z} \Bigr) \\
		%
		\Rightarrow^{\beta} \enskip &\Bigl( \lambda fz \ . \ f \ (f \ ( f \ (f z) \ ) \ ) \  )  \Bigr) = \num{4} 
	\end{align*}
\end{frame}
%
\begin{frame} \frametitle{Zusatzaufgabe 1}
	\textbf{Teil (b)} 
	\begin{equation*} 
		f(n) = s^n 
	\end{equation*}
	
	\bigskip
	
	\textbf{Teil (c)}
	\begin{equation*} 
		g(n,m) = m^n 
	\end{equation*}
	
	\begin{equation*} 
		\num{pow'} = \Bigl( \lambda n \colll{m} f z . n \colll{m} f z \Big) 
	\end{equation*}
\end{frame}

\end{document}

