\begin{exercisePage}
\begin{task}
	Wir betrachten die \textit{Burgers-Gleichung}
	\begin{equation*}
	\begin{aligned}
	u_t + uu_x &= 0 \qquad \text{für } (x,t) \in \R \times (0,\infty)\\
	u(x,0) &= g(x) \qquad\text{für alle } x \in \R
	\end{aligned}
	\end{equation*}
	\begin{enumerate}
		\item Es sei zuerst $g$ gegeben durch
		\begin{equation*}
			g(x) = \begin{cases} 1 &\text{für } x \geq 0,\\ 0 & \text{für } x < 0 \end{cases}
		\end{equation*}
		Zeigen Sie, dass die Funktionen
		\begin{equation*}
			v(x,t) = \begin{cases} 1 & \text{für } x \geq \tfrac12 t,\\ 0 &\text{für }x < \tfrac12t \end{cases}
			\qquad \und \qquad
			w(x,t) = \begin{cases} 1 & \text{für } x \geq t,\\ \frac{x}{t} & \text{für } x\in [0,t),\\ 0 & \text{für } x < 0 \end{cases}
		\end{equation*}
		schwache Lösungen sind.
		
		\textit{Hinweis}: Es genügt, die Randwerte, die Rankine-Hugoniot-Bedingung auf Sprungkurven und die Differentialgleichung abseits dieser Kurven zu überprüfen.
		
		\item Es sei nun $g$ gegeben durch
		\begin{equation*}
			g(x) = \begin{cases} 1 & \text{für } x\leq 0 \\ 1-x & \text{für } 0 \leq x \leq 1 \\ 0 & \text{für } x \geq 1 \end{cases}
		\end{equation*}
		\begin{enumerate}
			\item Offenbar ist $g \notin C^1(\R)$. Stellen Sie dennoch die charakteristischen Gleichungen auf und ermitteln Sie, welche Lösung man formal für $t<1$ erwarten würde.
			\item In welchen Punkten $(x,t)\in \R \times (0,1)$ gilt $u_t + u u_x = 0$?
			\item Setzen Sie $u$ derart auf $\R \times (0,\infty)$ fort, dass nur eine Sprungkurve existiert und dass entlang dieser die Rankine-Hugoniot-Bedingung erfüllt ist.
			
			\textit{Hinweis}: Es bietet sich an, dass $u$ für $t \geq 1$ nur die Werte $0$ und $1$ annimmt.
		\end{enumerate}
	\end{enumerate}
\end{task}

Für die Burgers-Gleichung gilt in Anlehnung an die Notation der Vorlesung $F(u)_x = u u_x$ und damit $F(u) = \frac{1}{2} u^2$.
\begin{enumerate}[label=(zu \alph*), leftmargin=*]
	\item Wir betrachten die Funktion $v$. Für die Randwerte, d.h. für $y=0$ gilt 
	\begin{equation*}
		v(x,0) = \begin{cases} 1 &\text{für } x \geq 0 \\ 0 & \text{für } x < 0 \end{cases} \enskip = g(x) \qquad \text{für alle } x \in \R
	\end{equation*}
	Die Sprungkurve ist gegeben durch $s(t) = \frac{1}{2} t$ mit $\dot{s} = \frac{1}{2}$. Es gilt $\dsq{v} = -1$ und $\dsq{F(v)} = 0 - \frac{1}{2} = \frac{1}{2}$, also ist mit $\dsq{F(v)} = \dot{s} * \dsq{v}$ die Rankine-Hugoniot-Bedingung erfüllt.
	Betrachte die Differentialgleichung abseits der Sprungkurve:
	\begin{itemize}
		\item Sei $x > \frac{1}{2} t$. Dann ist $v$ gegeben durch $v(x,t) = 1$ und somit $v_t = v_x \equiv 0$. Eingesetzt in die PDE ergibt dies $v_t + v v_x = 0 + 1 * 0 = 0$. $\checkmark$ 
		\item Sei $x < \frac{1}{2} t$. Dann ist $v$ gegeben durch $v(x,t) = 0$ und somit $v_t = v_x \equiv 0$. Eingesetzt in die PDE ergibt dies $v_t + v v_x = 0 + 0 * 0 = 0$. $\checkmark$ 
	\end{itemize}

	Betrachten wir nun die Funktion $w$. Die Randwerte werde wegen
	\begin{equation*}
		w(x,0) = \begin{cases} 1 &\text{für } x \geq 0 \\ 0 & \text{für } x < 0 \end{cases} \enskip = g(x) \qquad \text{für alle } x \in \R
	\end{equation*}
	erfüllt. Wir erhalten hier zwei Sprungkurven:
	\begin{itemize}
		\item Die Winkelhalbierende können wir durch $s_1(t) = t$ parametrisieren, also ist $\dot{s} = 1$. Dann ergibt sich für die Differenzen
		\begin{equation*}
			\dsq{F(w)} = \frac{1}{2} - \frac{1}{2} = 0  \quad \und \quad \dsq{w} = 1 - \frac{t}{t} = 0 \quad \follows \quad \dsq{F(w)} = \dot{s_1} * \dsq{w}
		\end{equation*}
		\item Für die andere Sprungkurve, die wir mit $s_2(t) = 0$ parametrisieren ($\dot{s} = 0$), erhalten wir 
		\begin{equation*}
		\dsq{F(w)} = 0 - 0 = 0  \quad \und \quad \dsq{w} = \frac{0}{t} - 0 = 0 \quad \follows \quad \dsq{F(w)} = \dot{s_2} * \dsq{w}
		\end{equation*}
	\end{itemize}
	Die Differentialgleichung wird abseits der Sprungkurven auch erfüllt:
	\begin{itemize}
		\item Sei $x > t$. Dann ist $w$ gegeben durch $w(x,t) = 1$ und somit $w_t = w_x \equiv 0$. Eingesetzt in die PDE ergibt dies $w_t + w w_x = 0 + 1 * 0 = 0$. $\checkmark$ 
		\item Sei $x \in [0,t)$. Dann ist $w$ gegeben durch $w(x,t) = \frac{x}{t}$ und somit $w_t = \frac{- x}{t^2}$ und $w_x = \frac{1}{t}$. Eingesetzt in die PDE ergibt dies $w_t + w w_x = \frac{- x}{t^2} + \frac{x}{t} \frac{1}{t} = 0$. $\checkmark$ 
		\item Sei $x < 0$. Dann ist $w$ gegeben durch $w(x,t) = 0$ und somit $w_t = w_x \equiv 0$. Eingesetzt in die PDE ergibt dies $w_t + w w_x = 0 + 0 * 0 = 0$. $\checkmark$ 
	\end{itemize}

	Damit sind also $v$ und $w$ schwache Lösungen der partiellen Differentialgleichung.
	
	\item Mit $y \defeq (x,t)$, $a(u,y) = \left( \begin{smallmatrix} u \\ 1 \end{smallmatrix} \right)$ und $b(u,y) = 0$ hat die Burgergsgleichung die quasilineare Form der Vorlesung. Dann sind die charakteristischen Gleichungen gegeben durch
	\begin{equation*}
		\begin{aligned}
			\dot{y}(\tau, \sigma) &= \left( \begin{smallmatrix} \dot{x} \\ \dot{t} \end{smallmatrix} \right) = a(\alpha, y) = \left( \begin{smallmatrix} \alpha \\ 1 \end{smallmatrix} \right) \\
			\dot{\alpha}(\tau, \sigma) &= a(\alpha, y) * Du = - b(u,y) = 0 \enskip \mit \enskip \alpha(0,\sigma) = g(\sigma)
		\end{aligned}
	\end{equation*}
	Aus der zweiten Gleichung erhalten wir die Lösung $\alpha(\tau, \sigma) = g(\sigma)$, d.h. $u$ ist entlang der Charakteristiken konstant. Aus der ersten Gleichung erhalten wir zum einen die Identifizierung $\tau = t$, d.h. die Charakteristiken können durch die Zeit $t$ parametrisiert werden. Zum anderen die gewöhnliche Differentialgleichung $\dot{x} = \alpha$ mit der Lösung $x(\tau, \sigma) = g(\sigma) * \tau + s$ bzw. mit der Identifizierung dann $x(t) = g(\sigma) * t + \sigma$. Setzen wir die Definition von $g$ ein, so erhalten wir
	\begin{equation*}
		x(t) = \begin{cases}
			t + \sigma       & \sigma \le 0 \\
			(1-\sigma) t + \sigma & 0 \le \sigma \le 1 \\
			\sigma           & \sigma \ge 1
		\end{cases}
	\end{equation*}
	Für $t \le 1$ können wir jeden Fall umstellen und erhalten
	\begin{equation*}
		\sigma = \begin{cases}
			x - t & x - t \le 0\\
			\frac{x-t}{1-t} & x-t \ge 0 \land x \le 1 \\
			x & x \ge 1
		\end{cases}
	\end{equation*}
	Nach Konstruktion erhalten wir dann die Lösung
	\begin{equation*}
		u(x,t) = \alpha(\tau,\sigma) = g(\sigma(t,x)) = \begin{cases}
			1                   & x - t \le 0 \\
			1 - \frac{x-t}{1-t} & x-t \ge 0 \land x \le 1 \\
			0                   & x \ge 1
		\end{cases} 
	\end{equation*}
	Sei nun $t \in (0,1)$. Wir prüfen die Gültigkeit der Differentialgleichung:
	\begin{itemize}
		\item Sei $x - t \le 0$.  Dann ist also $u(x,t) = 1$ und $u_x = u_t \equiv 0$ und die Differentialgleichung erfüllt. 
		\item Ist $x-t \ge 0$ und $x \le 1$, dann ist $u(x,t) = 1 - \frac{x-t}{1-t}$ und $u_x(x,t) = \frac{1}{1-t}$ sowie $u_t(x,t) = \frac{(t-1)+(x-t)}{(1-t)^2}$. In die Differentialgleichung eingesetzt ergibt dies
		\begin{equation*}
			u_t + u u_x = \frac{(t-1)+(x-t)}{(1-t)^2} + \frac{1}{1-t} - \frac{x-t}{(1-t)^2} = 0 \quad \checkmark
		\end{equation*}
		\item Sei $x \ge 1$. Dann ist $u = u_x = u_t \equiv 0$ und die Differentialgleichung damit erfüllt.
	\end{itemize}
	Somit ist $u$ Lösung der Differentialgleichung für alle $x \in \R$ und $t \in (0,1)$.
\end{enumerate}



\begin{task}
	Betrachten Sie das Beispiel \enquote{Ampel (von rot auf grün)} aus der Vorlesung, modelliert durch
	\begin{equation*}
		\begin{aligned}
			u_t + F(u)_x &= 0 \qquad \text{für } (x,t) \in \R \times (0,\infty) \\
			u(x,0) &= g(x) \qquad \text{für alle } x \in \R
		\end{aligned}
	\end{equation*}
	mit 
	\begin{equation*}
		F(u) = u(60 - \frac{2}{5} u)\quad \text{und} \quad g(x) = \begin{cases} 150 & \text{für } x<0 \\ 0 & \text{für } x \ge 0\end{cases}
	\end{equation*}
	\begin{enumerate}
		\item In der Vorlesung wurde in Teil a) eine Funktion $u_a$ explizit gegeben. In Teil b) wurde eine weitere Lösung $u_b$ skizziert. Ermitteln Sie $u_b$ explizit.
		
		\emph{Hinweis}: Die charakteristischen Gleichungen geben $u_b$ auf eine großen Menge vor. Ermitteln Sie $u_b$ für die restlichen $(x,t)$, indem sie eine differenzierbare reelle Funktion $v$ derat bestimmen, dass $u_b(x,t) = v(\frac{x}{t})$ die Differentialgleichung löst.
		
		\item Überprüfen Sie für $u_a$ und $u_b$ entlang aller Sprungkurven die Rankine-Hugoniot-Bedingung und die Entropiebedingung.
	\end{enumerate}
\end{task}

\begin{enumerate}[label=(zu \alph*)]
	\item Für die skalare Erhaltungsgleichung erhalten wir die aus der Vorlesung bekannten Bezeichnungen mit $y = (x,t)$, $a(u,y) = (60 - \frac{4}{5} u \ , \ 1)$ und $b \equiv 0$. Außerdem sind die Startwerte der charakteristischen Gleichungen gegeben durch $x_0(s) = s$, $t_0(s) = 0$, und als Randwerte $g(s) = 150 * \one_{\menge{s < 0}}(s)$. Wir erhalten die charakteristischen Gleichungen
	\begin{equation*}
		\begin{aligned}
			\dot{y}(\tau, \sigma) &= \begin{pmatrix} \dot{x} \\ \dot{t} \end{pmatrix} = a(\alpha, y) = \begin{pmatrix} 60 - \frac{4}{5} u \\ 1 \end{pmatrix} \\
			\dot{\alpha}(\tau, \sigma) \overset{\text{DGl}}&{=} - b(\alpha, y) = 0  
		\end{aligned}
	\end{equation*}
	Aus der letzten Zeile der ersten Gleichung erhalten wir dabei die Identifizierung $t(\tau, \sigma) = \tau$. Die zweite charakteristische Gleichung löst sich unter Nutzung des Anfangswertes $\alpha(0,\sigma) = g(\sigma)$ zu $\alpha(\tau, \sigma) = g(\sigma)$. Setzen wir dies in die erste Zeile der ersten Gleichung ein, so erhalten wir $\dot{x}(\tau, \sigma) = 60 - \frac{4}{5} g(\sigma)$. Dies löst sich mit dem Anfangswert $x(0,\sigma) = \sigma$ zu 
	\begin{equation*}
		x(\tau, \sigma) = 60\tau - \frac{4}{5} g(\sigma) * \tau + \sigma = 
		\begin{cases}
			-60 t + \sigma & \sigma < 0 \\
			 60 t + \sigma & \sigma \ge 0
		\end{cases}
	\end{equation*}
	Für $\tau =  t > 0$ lässt sich dies umstellen zu
	\begin{equation*}
		\sigma(x,t) = 
		\begin{cases}
			x + 60 t & x < 0 \\
			x - 60 t & x \ge 0
		\end{cases}
	\end{equation*}
	Damit erhalten wir die Lösung 
	\begin{equation*}
		u(x,t) = \alpha(t, \sigma) = 
		\begin{cases}
			150 & \sigma(x,t) < 0 \\
			0   & \sigma(x,t) \ge 0
		\end{cases}
	\end{equation*}
	Da $t \in (0, \infty)$ ist, ist somit $u$ auf der Menge $\menge{(x,t) : t \in (0,\infty), x \notin (-60t, 60t)}$ eindeutig vorgegeben. Wir erweitern nun auf $t \in \R$ und betrachte $x \in (-60t, 60t)$. Dabei soll eine Funktion $v$ mit $u(x,t) = v(\frac{x}{t})$ bestimmt werden, die die PDE erfüllt. Die Funktion $u$ muss gemäß Vorlesung ihr Maximum auf der $t$-Achse, also für $x=0$ annehmen. Dementsprechend gilt $u(0,t) = v(\frac{0}{t}) = v(0) = u_\text{opt} = 75$. Da wir $x \in (-60t, 60t)$ betrachten, gilt $\frac{x}{t} = (-60,60)$. Die Randwerte sollten dabei mit minimaler (rechts) bzw. maximaler (links) Dichte gegeben sein, d.h. $v(-60) = 150$ und $v(60) = 0$. Diese drei Punkte definieren uns eine eindeutige Polynomfunktion zweiten Grades:
	\begin{equation*}
		v(\xi) = -\frac{5}{4s} \xi + 75 \qquad \text{auf } (-60, 60)
	\end{equation*}
	Damit können wir $u$ definieren als
	\begin{equation*}
		u(x,t) = v(\frac{x}{t}) = -\frac{5}{4} \frac{x}{t} + 75 \qquad \text{für } t \in \R \und x \in (-60t, 60t)
	\end{equation*}
	Damit gilt $u_x(x,t) = -\frac{5}{4} \frac{1}{t}$ und $u_t(x,t) = \frac{5}{4} \frac{x}{t^2}$. In die PDE eingesetzt liefert dies
	\begin{equation*}
		\begin{aligned}
			u_t + F(u)_x &= u_t + \brackets{60 - \frac{4}{5} u} u_x \\
			&= \frac{5}{4} \frac{x}{t^2} + \brackets{60 + \frac{x}{t} - 60} \brackets{-\frac{5}{4} \frac{1}{t}} 
			= \frac{5}{4} \frac{x}{t^2} -\frac{5}{4} \frac{x}{t^2} 
			= 0 \qquad \checkmark
		\end{aligned}
	\end{equation*} 
	Somit können wir $u$ schlussendlich definieren als
	\begin{equation*}
		u(t,x) = 
		\begin{cases}
			150 & x < -60t \\
			-\frac{5}{4} \frac{x}{t} + 75 & -60t \le x \le 60t \\
			0 & x > 60t
		\end{cases}
	\end{equation*}
	\item Definiere $u_b \defeq u$ von oben. Man sieht leicht, dass $u_b$ eine stetige Funktion ist. Daher existieren keine Sprungkurven und die Rankine-Hugoniot- bzw. Entropie-Bedingungen müssen nicht betrachtet werden. (Die Rankine-Hugoniot-Bedingungen werden trivialerweise trotzdem erfüllt, aber die Entropie-Bedingung nicht mehr).
	
	Aus der Vorlesung bekannt ist 
	\begin{equation*}
		u_a(x,t) = 
		\begin{cases}
			150 & x < 0 \\
			0   & x \ge 0
		\end{cases}
	\end{equation*}
	Wir betrachten die Sprungkurve $x=s(t)=0$. Dann ist auch $\dot{s} = 0$ und mit $F(u) = u \brackets{60 - \frac{2}{5} u}$ gilt
	\begin{equation*}
		\left.\begin{array}{rcl}
			\dsq{u} &=& 150 \\
			\dsq{F(u)} &=& 0
		\end{array}
		\right\} \follows \dsq{F(u)} = \dot{s} * \dsq{u}
	\end{equation*}
	Es ist $F'(u) = 60 - \frac{4}{5} u$ und damit $F'(u_l) = F'(150) = -60$ sowie $F'(u_r) = F'(0) = 60$. Somit ist die Entropie-Bedingung verletzt.
\end{enumerate}

\end{exercisePage}