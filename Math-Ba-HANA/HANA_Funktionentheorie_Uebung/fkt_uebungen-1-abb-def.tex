\documentclass[margin=5mm]{standalone}
\usepackage{tudscrcolor} 
\usepackage{pgfplots}
\usepgfplotslibrary{fillbetween} 
\pgfplotsset{
  compat=1.10,% mit writeLaTeX bisher noch nicht möglich
  flaeche/.style={draw=none,fill=black,fill opacity=0.2},
  every axis/.append style={
  	axis x line=middle,    % put the x axis in the middle
  	axis y line=middle,    % put the y axis in the middle
  	axis line style={->}, % arrows on the axis
  	xlabel={$\Re(z)$},          % default put x on x-axis
  	ylabel={$\Im(z)$},          % default put y on y-axis
  	enlargelimits=0.05,
  	grid,
  	domain=-5:5,
  	ymin = -5,
  	ymax = 5,
  	xmin = -5,
  	xmax = 5,
  }
}

\renewcommand{\Re}{\mathrm{Re}}
\renewcommand{\Im}{\mathrm{Im}}



\begin{document} 
	
	
%%%%%%%%%%%%%%%%%%%%%%%%%%%%%%%%%%%%%%%%%%%%%%%%%%%%%%%%%%%%%%%%%%%%%%%%%%%%%%%%%%%%%%%%%%%%%%%%%%%%%%%
% Teil (d)

\begin{tikzpicture} 
\begin{axis}[%
	xmin = -1.5,
	xmax =  1.5,
	ymin = -1.5,
	ymax =  1.5,
	axis equal
]
	\addplot [domain=0:360,samples=50, variable=\x]({cos(x)}, {sin(x)});
	
%	\node[label={0:{$z_0$}},circle,fill,inner sep=1.5pt, cddarkblue] at (axis cs:0:1cm) {};
%	\node[data cs=polar,red, inner sep=1.5pt] at (axis xs: 1,{0});
%	\node[label={0:{$z_2$}},circle,fill,inner sep=1.5pt, cdpurple] at (axis cs:2.5,2.5) {};
	\addplot[cddarkblue,line width=1pt, mark=*] coordinates {(1.0, 0.0)
		( 0.3090,  0.9510)
		(-0.8090,  0.5877)
		(-0.8090, -0.5877)
		( 0.3090, -0.9510)
		(1.0, 0.0)};
	\node[label={-20:{$z_0$}}, draw=none] at (axis cs:1,0) {};
	\node[label={ 20:{$z_1$}}, draw=none] at (axis cs:0.3090,  0.9510) {};
	\node[label={160:{$z_2$}}, draw=none] at (axis cs:-0.8090,  0.5877) {};
	\node[label={200:{$z_3$}}, draw=none] at (axis cs:-0.8090, -0.5877) {};
	\node[label={270:{$z_4$}}, draw=none] at (axis cs:0.3090, -0.9510) {};
\end{axis} 
\node[above,font=\large\bfseries] at (current bounding box.north) {Teil (d): $z^5 = 1$};
\end{tikzpicture} 







%%%%%%%%%%%%%%%%%%%%%%%%%%%%%%%%%%%%%%%%%%%%%%%%%%%%%%%%%%%%%%%%%%%%%%%%%%%%%%%%%%%%%%%%%%%%%%%%%%%%%%%
% Teil (e)

\begin{tikzpicture} 
\begin{axis}[%
	xmin = -3,
	xmax =  10,
	ymin = -2,
	ymax =  5,
	axis equal
]
	
	\addplot [domain=0:3.14, smooth,, samples=100, variable=\x, cddarkblue, line width=2pt]({3 + 5 * cos(deg(x)))}, {-1 + 5 * sin(deg(x)))});
	
\end{axis} 
\node[above,font=\large\bfseries] at (current bounding box.north) {Teil (e): $z = 3 - \mathrm{i} + 5e^{\mathrm{i}t} \enskip (0 \le t \le \pi)$};
\end{tikzpicture} 






%%%%%%%%%%%%%%%%%%%%%%%%%%%%%%%%%%%%%%%%%%%%%%%%%%%%%%%%%%%%%%%%%%%%%%%%%%%%%%%%%%%%%%%%%%%%%%%%%%%%%%%
% Teil (f)

\begin{tikzpicture} 
\begin{axis}[%
	xmin = -30,
	xmax =  30,
	ymin = -30,
	ymax =  30,
	axis equal
]
\addplot [domain=0:25, smooth,, samples=100, variable=\x, cddarkblue, line width=2pt]({rad(deg(x) * cos(deg(x)))}, {rad(deg(x) * sin(deg(x)))});

\end{axis} 
\node[above,font=\large\bfseries] at (current bounding box.north) {Teil (f): $z = t e^{\mathrm{i}t} \enskip (t \ge 0)$};
\end{tikzpicture} 

\end{document}