\documentclass{beamer}
\usepackage{../tut-slides}
\usepackage{../mathoperatorsAuD}

\usepackage{csquotes}
\usepackage{cancel}

\usepackage{amsmath,amssymb}

\usepackage{tikz}
\usetikzlibrary{positioning,automata, matrix, trees}
\usetikzlibrary{calc,positioning,backgrounds,arrows.meta}
\usepackage{forest}


\usepackage{booktabs}
\usepackage{tabularx}
\usepackage{tabu}

\renewcommand{\tabularxcolumn}[1]{>{\hspace{0pt}}m{#1}}

\usepackage{listings}
\lstset{ 
	basicstyle=\footnotesize\ttfamily,        % the size of the fonts that are used for the code
	breakatwhitespace=false,         % sets if automatic breaks should only happen at whitespace
	breaklines=true,                 % sets automatic line breaking
	commentstyle=\itshape,    	     % comment style
	escapeinside={\%*}{*)},          % if you want to add LaTeX within your code
	extendedchars=true,              % lets you use non-ASCII characters; for 8-bits encodings only, does not work with UTF-8
	firstnumber=1,                % start line enumeration with line 1000
	frame=single,
	keywordstyle=\bfseries,       % keyword style
	morekeywords={}, 
	language=Prolog,                 % the language of the code
	numbers=left,                    % where to put the line-numbers; possible: (none, left, right)
	numbersep=5pt,                   % how far the line-numbers are from the code
	numberstyle=\tiny\color{cdgray!50}, % the style that is used for the line-numbers
	rulecolor=\color{cddarkblue}, 
	tabsize=2,	                   % sets default tabsize to 2 spaces
}
\lstdefinestyle{refutation}{ 
	basicstyle=\footnotesize\ttfamily,        % the size of the fonts that are used for the code
	breakatwhitespace=false,         % sets if automatic breaks should only happen at whitespace
	breaklines=true,                 % sets automatic line breaking
	commentstyle=\color{cdgray},    	     % comment style
	escapeinside={(*@}{@*)},          % if you want to add LaTeX within your code
	extendedchars=true,              % lets you use non-ASCII characters; for 8-bits encodings only, does not work with UTF-8
	firstnumber=1,                % start line enumeration with line 1000
	frame=none,
	keywordstyle=\bfseries,       % keyword style
	morekeywords={}, 
	language=Prolog,                 % the language of the code
	numbers=none,                    % where to put the line-numbers; possible: (none, left, right)
	numbersep=5pt,                   % how far the line-numbers are from the code
	numberstyle=\tiny\color{cdgray!50}, % the style that is used for the line-numbers
	rulecolor=\color{cddarkblue}, 
	tabsize=2,	                   % sets default tabsize to 2 spaces
}


\usepackage{textgreek}


\renewcommand{\emph}[1]{\textbf{#1}}
\newcommand{\coloremph}[1]{\textcolor{cdpurple}{#1}}
\newcommand{\col}[1]{\textcolor{cdpurple}{\boldsymbol{#1}}}
\newcommand{\coll}[1]{\textcolor{cddarkgreen}{\boldsymbol{#1}}}
\newcommand{\colll}[1]{\textcolor{cdorange}{\boldsymbol{#1}}}
%\newcommand{\step}[2][]{\ensuremath{\overset{{#1} (\text{#2})}{=}}}
%\newcommand*{\astep}[2][]{\ensuremath{\overset{{#1} (\text{#2})}&{=}}}

\newcommand{\num}[1]{\ensuremath{\langle #1 \rangle}}

\begin{document}	
	\title{Programmierung}
	\subtitle{Übung 9: Logikprogrammierung mit Prolog${}^-$}
	\author{Eric Kunze}
	\email{eric.kunze@mailbox.tu-dresden.de}
%	\city{TU Dresden}
	\date{}
%	\institute{Lehrstuhl für Grundlagen der Programmierung}
%	\titlegraphic{\includegraphics[width=2cm]{../TUD-white.pdf}}
	
	\maketitle
	

%%%%%%%%%%%%%%%%%%%%%%%%%%%%%%%%%%%%%%%%%%%%%%%%%%%%%%%%%%%%%%%%%%%%%%%%%%%%%

\begin{frame}[fragile] \frametitle{Inhalt}
	\begin{enumerate}
		\item Funktionale Programmierung
		\begin{enumerate}
			\item Einführung in Haskell: Listen
			\item Algebraische Datentypen
			\item Funktionen höherer Ordnung
			\item Typpolymorphie \& Unifikation
			\item Beweis von Programmeigenschaften
			\item \textlambda--Kalkül
		\end{enumerate}
		\item \textbf{Logikprogrammierung}
		\item Implementierung einer imperativen Programmiersprache
		\item Verifikation von Programmeigenschaften
		\item H${}_\text{0}$ -- ein einfacher Kern von Haskell
	\end{enumerate}
\end{frame}


\section{Logikprogrammierung und Prolog${}^-$}

\begin{frame} \frametitle{Einführung in Prolog}
	\small
	\begin{itemize}
		\item Französisch: programmation en logique \\
		(deutsch: Programmieren in Logik)
		\item \emph{Online-Editor \& Interpreter:} \url{swish.swi-prolog.org}
		\item Prolog-Programme bestehen aus \emph{Fakten} und \emph{Regeln}.
		\bigskip
		\item Statements werden mit \emph{\texttt{.}} abgeschlossen.
		\item Variablen beginnen mit Großbuchstaben.
		\bigskip
		\item \emph{UND}-Operator. \hspace{.2cm} \texttt{,}
		\item \emph{ODER}-Operator.\hspace{.2cm} \texttt{;}
	\end{itemize}
\end{frame}


%%%%%%%%%%%%%%%%%%%%%%%%%%%%%%%%%%%%%%%%%%%%%%%%%%%%%%%%%%%%%%%%%%%%%%%%%%%%%%%%


\section{Aufgabe 1}


\begin{frame}[fragile] \frametitle{Aufgabe 1 -- Teil (a)}
	
	\textbf{Ziel:} binäre Relation \texttt{sublist} mit $(\ell_1 , \ell_2) \in \texttt{sublist} \Leftrightarrow \ell_1 \subseteq \ell_2$ \\
	$\to$ $\ell_1$ soll \textit{Teilliste} von $\ell_2$ sein
	
	\begin{lstlisting}
nat (0).
nat(s(X)) :- nat(X).

listnat ([]).
listnat ([X|XS]) :- nat(X), listnat(XS).

	\end{lstlisting}
	\pause
	\begin{lstlisting}[firstnumber=6]
sublist(Xs   , [Y|Ys]) :- nat(Y), sublist(Xs, Ys). 
sublist(Xs   , Ys )    :- prefix(Xs, Ys). 

prefix([]    , Ys )    :- listnat(Ys). 
prefix([X|Xs], [X|Ys]) :- nat(X), prefix(Xs, Ys).
	\end{lstlisting}
\end{frame}

\begin{frame}[fragile] \frametitle{Aufgabe 1 -- Teil (b)}
	\textbf{Belegung 1:}
	
	\begin{lstlisting}[style=refutation, basicstyle=\ttfamily\scriptsize]
          ?-  sublist ([<4>|Xs], [<5>, <4>, <3>]).
          ?-  nat(<5>), sublist ([<4>|Xs], [<4>, <3>]). % 6
          ?-* nat(0), sublist ([<4>|Xs], [<4>, <3>]).   % 2
          ?-  sublist ([<4>|Xs], [<4>, <3>]).           % 1
          ?-  prefix ([<4>|Xs], [<4>, <3>]).            % 7
          ?-  nat(<4>), prefix(Xs , [<3>]).             % 10
          ?-* nat(0), prefix(Xs , [<3>]).               % 2
          ?-  prefix(Xs , [<3>]).                       % 1
{Xs = []} ?-  listnat ([<3>]).                          % 9
          ?-  nat(<3>), listnat ([]).                   % 5
          ?-* nat(0), listnat ([]).                     % 2
          ?-  listnat ([]).                             % 1
          ?-  .                                         % 4
	\end{lstlisting}
	Somit also \texttt{Xs = []}.
\end{frame}

\begin{frame}[fragile] \frametitle{Aufgabe 1 -- Teil (b)}
	\textbf{Belegung 2:}
	
	\begin{lstlisting}[style=refutation, basicstyle=\ttfamily\scriptsize]
               ?-  sublist ([<4>|Xs], [<5>, <4>, <3>]).
               ?-  nat(<5>), sublist ([<4>|Xs], [<4>, <3>]). 
                                                        % 6
               ?-* nat(0), sublist ([<4>|Xs], [<4>, <3>]).  
                                                        % 2
               ?-  sublist ([<4>|Xs], [<4>, <3>]).      % 1
               ?-  prefix ([<4>|Xs], [<4>, <3>]).       % 7
               ?-  nat(<4>), prefix(Xs , [<3>]).        % 10
               ?-* nat(0), prefix(Xs , [<3>]).          % 2
               ?-  prefix(Xs , [<3>]).                  % 1
{Xs=[<3>|Xs1]} ?-  nat(<3>), prefix(Xs1 , []).          % 10
               ?-* nat(0), prefix(Xs1 , []).            % 2
               ?-  prefix(Xs1 , []).                    % 1
{Xs1 = []}     ?-  listnat ([]).                        % 9
               ?-  .                                    % 4
	\end{lstlisting}
	
	Somit also \texttt{Xs = [<3>|Xs1] = [<3>]}.
\end{frame}


%%%%%%%%%%%%%%%%%%%%%%%%%%%%%%%%%%%%%%%%%%%%%%%%%%%%%%%%%%%%%%%%%%%%%%%%%%%%%%%%


\section{Aufgabe 2}

\begin{frame}[fragile] \frametitle{Aufgabe 2}
	Wir wollen einen binären Termbaum auswerten. 
	\begin{lstlisting}
nat (0).
nat(s(X)) :- nat(X).
sum(0, Y, Y) :- nat(Y).
sum(s(X), Y, s(S)) :- sum(X, Y, S).
	\end{lstlisting}
	
	\textbf{Beispiel:} 
	
	\begin{forest}
		[\texttt{minus} [\texttt{<4>}] [\texttt{plus} [\texttt{<2>}] [\texttt{<1>}]]]
	\end{forest}
\end{frame}

\begin{frame}[fragile] \frametitle{Aufgabe 2}
	Wir wollen einen binären Termbaum auswerten. 
	\begin{lstlisting}
nat (0).
nat(s(X)) :- nat(X).
sum(0, Y, Y) :- nat(Y).
sum(s(X), Y, s(S)) :- sum(X, Y, S).
	\end{lstlisting}
	
	\begin{lstlisting}[firstnumber=5, basicstyle=\ttfamily\tiny]
eval( X         , X ) :-  nat(X). 
eval( plus (L,R), X ) :-  eval(L, LE), eval(R, RE), sum(LE, RE,  X). 
eval( minus(L,R), X ) :-  eval(L, LE), eval(R, RE), sum(RE,  X, LE).
	\end{lstlisting}
\end{frame}


%%%%%%%%%%%%%%%%%%%%%%%%%%%%%%%%%%%%%%%%%%%%%%%%%%%%%%%%%%%%%%%%%%%%%%%%%%%%%%%%


\section{Aufgabe 3}

\begin{frame}[fragile] \frametitle{Aufgabe 3 -- Teil (a)}
	\begin{lstlisting}
subt( X  , X         ). 
subt( S1 , s(_ , T2) ) :- subt(S1,T2).
subt( S1 , s(T1, _ ) ) :- subt(S1,T1).
	\end{lstlisting}
	
	
	\begin{lstlisting}[style=refutation, basicstyle=\ttfamily\scriptsize]
                        ?-  subt(s(X,Y), s(s(a,b), s(b,a))). 
{X = s(a,b), Y=s(b,a)}  ?-  .                       % 1  

                        ?-  subt(s(X,Y), s(s(a,b), s(b,a))).
                        ?-  subt(s(X,Y), s(b,a)).   % 2
{X = b, Y=a}            ?-  .                       % 1

                        ?-  subt(s(X,Y), s(s(a,b), s(b,a))).
                        ?-  subt(s(X,Y), s(a,b)).   % 3
{X = a, Y=b }           ?-  .                       % 1
	\end{lstlisting}
\end{frame}

\begin{frame}[fragile] \frametitle{Aufgabe 3 -- Teil (b)}
	\begin{lstlisting}
subt( X  , X         ). 
subt( S1 , s(_ , T2) ) :- subt(S1,T2).
subt( S1 , s(T1, _ ) ) :- subt(S1,T1).
	\end{lstlisting}


\begin{lstlisting}[style=refutation, basicstyle=\ttfamily\scriptsize]
                  ?-  subt(s(a,a), X). 
{X = s(a,a)}      ?-  .                  % 1
                                         (*@\centering$\Rightarrow$@*) X = s(a,a))
                  ?-  subt(s(a,a), X). 
{X = s(_ , X1)}   ?-  subt(s(a,a), X1).  % 2 
{X1 = s(a,a)}     ?-  .                  % 1 
                                         (*@\centering$\Rightarrow$@*) X = s(a,s(a,a))

                  ?-  subt(s(a,a), X). 
{X = s(X2, _)}    ?-  subt(s(a,a), X2).  % 3 
{X2 = s(a,a) }    ?-  .                  % 1 
                                         (*@\centering$\Rightarrow$@*) X = s(s(a,a),c)
\end{lstlisting}
\end{frame}

\end{document}

